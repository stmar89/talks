% \documentclass{beamer}
\documentclass[handout]{beamer}


\usetheme{AnnArbor}
% \usetheme{Pittsburgh}
% \usecolortheme{whale}
% \usecolortheme{wolverine}
\usepackage{commands}
\usepackage{faktor}
\usepackage{xfrac} 

\renewcommand{\char}{char}

%AUTHOR DETAILS
%%%%%%%%%%%%%%%%%%%%%%%%%%%%%%%%%%%%%%%%%%%%%%%%
\title[]{Computing isomorphism classes of abelian varieties over finite fields}
\author[Marseglia Stefano]{Marseglia Stefano}
\institute[]{MPI/Stockholms University}
\date{11 October  2018}

\begin{document}

\begin{frame}
\titlepage
\end{frame}

\begin{frame}{ Introduction }
\begin{df}
 An \textbf{abelian variety} over a field $k$ is a complete connected group variety over $k$.
\end{df}
 \pause eg: 
 AV's of dimension $1$ are elliptic curves.
 \[ Y^2Z=X^3+AXZ^2+BZ^3 \qquad 4A^3+27B^2\neq0 \]
\end{frame}

\begin{frame}{ Abelian varieties ($\C$ vs $ \F_q $) }
\begin{itemize}
 \item Goal: compute \textbf{isomorphism classes} of abelian varieties over a \textbf{finite field} $\F_q$.
 \pause \item in dimension $g>1$ is not easy to produce equations.
 \pause \item for $g>3$ it is not enough to consider Jacobians.
 \pause \item over $\C$:
 \[
      \set{ \text{abelian varieties $/\C$} } \longleftrightarrow 
      \set{\parbox[c]{8em}{$\C^g/L$ with $L\simeq \Z^{2g}$\\ $+$ Riemann form}}.
 \]
 \pause \vspace{-6mm} \item in positive characteristic we don't have such equivalence (on the whole category).
\end{itemize}
\end{frame}

\begin{frame}{ Deligne's equivalence }
\begin{theorem}[Deligne '69]
Let $q=p^r$, with $p$ a prime. There is an equivalence of categories:
\[\begin{array}{cc}
\set{\text{\textbf{Ordinary} abelian varieties over $\F_q$}}	& A \\
\pause \updownarrow							& \downmapsto \\
\set{\parbox[p]{19em}{pairs $(T,F)$, where $T\simeq_\Z \Z^{2g}$ and $T\overset{F}{\to} T$ s.t.\\
- $F\otimes \Q$ is semisimple\\
- the roots of $\Char_{F\otimes\Q}(x)$ have abs. value $\sqrt{q}$\\
- \textbf{half of them are $p$-adic units}\\
- $\exists V:T\to T$ such that $FV=VF=q$
}}	& (T(A),F(A))
\end{array}\]
\end{theorem}
\pause
\begin{remark}
\begin{itemize}
 \item If $\dim(A)=g$ then $\rk(T(A))=2g$;
 \item $\Frob(A)\rightsquigarrow F(A)$.
\end{itemize}
\end{remark}
\end{frame}

\begin{frame}{ Deligne's equivalence: square-free case}
Fix a \textbf{ordinary square-free} characteristic $q$-Weil polynomial $h$.\\
\[\rightsquigarrow \text{an isogeny class $\cC_h$ (by Honda-Tate)}.\]
\pause Put 
\[K := \Q[x]/(h) \text{ and } F:= x \mod h. \]
\pause Deligne's equivalence induces:
\begin{theorem}[M.]
\vspace{-5mm}\[\begin{array}{cc}
\faktor{\set{\text{Ordinary abelian varieties over $\F_q$ in $\cC_h$}}}{\simeq} & \\
\updownarrow & \\
\faktor{\set{ \text{fractional ideals of $\Z[F,q/F] \subset K$ } }}{\simeq} & =:\textcolor{blue}{\ICM(\Z[F,q/F])}\\ 
  & \textcolor{blue}{\text{ ideal class monoid} }
  \end{array}\]
\end{theorem}
\end{frame}

\begin{frame}{ICM : Ideal Class Monoid}
Let $R$ be an order in a finite \'etale  $\Q$-algebra $K$ (with ring of integers $\cO_K$).\\
% according to Bourbaki etale (over a field K) implies commutative and finite (since it is defined as being isomorphic to L^n for some extension L of K).
\pause Recall: for fractional $R$-ideals $I$ and $J$
\[ I\simeq_R J \Longleftrightarrow \exists x \in K^\times \text{ s.t.~} xI=J \]
\pause Define the \textbf{ideal class monoid} of $R$ as
      \[\ICM(R) := \faktor{\set{\text{fractional $R$-ideals}}}{\simeq_R} \pause\textcolor{blue}{\quad\supseteq \Pic(R)}\]
\pause To compute $\ICM(R)$:
\begin{enumerate}[(1)]
\pause \item tackle the problem \textbf{locally} at every $\p$ of $R$,
\pause \item then consider the action of the \textbf{invertible} ideals.
\end{enumerate}
\end{frame}

\begin{frame}{ back to AV's: Dual variety/Polarization }
\begin{itemize}
 \item Howe ('95) defined a notion of \textbf{dual} module and of \textbf{polarization} in the category of Deligne modules.
\pause \item Concretely, if $A\leftrightarrow I$, then $A^\vee \leftrightarrow \overline{I}^t$, and
\pause \item a polarization $\mu$ of $A$ corresponds to a $\lambda\in K^\times$ such that
      \begin{enumerate}[-]
       \item $\lambda I \subseteq \overline{I}^t$ (isogeny);
       \item $\lambda$ is totally imaginary ($\overline \lambda = -\lambda$);
       \item $\lambda$ is $\Phi$-positive, where $\Phi$ is a \textcolor{red}{specific} CM-type of $K$. \textcolor{red}{\parbox{6em}{\center "coming from char $p$"}}
      \end{enumerate} 
      Also: $\deg \mu= [\overline{I}^t : \lambda I]$.
\pause  \item if $A \leftrightarrow I$ and $S=(I:I)$ then
  \[\set{\parbox[p]{7.5em}{non-isomorphic polarizations of $A$}} \longleftrightarrow \dfrac{\set{\text{totally positive }u\in S^\times }}{\set{v\overline{v}: v\in S^\times}}\]
  and $\Aut(A,\mu) = \set{\text{torsion units of $S$}}$
\end{itemize}
\end{frame}

\begin{frame}{ Example}
\begin{itemize}
 \item Let $h(x)=x^8 - 5x^7 + 13x^6 - 25x^5 + 44x^4 - 75x^3 + 117x^2 - 135x + 81$;
 \item $\rightsquigarrow$ isogeny class of an simple ordinary abelian varieties over $\F_{3}$ of dimension $4$;
 \item Let $F$ be a root of $h(x)$ and put $R:=\Z[F,3/F]\subset \Q(F)$;
 \item $8$ over-orders of $R$: two of them are not Gorenstein;
 \item $\#\ICM(R) = 18 \rightsquigarrow 18$ isom.~classes of AV in the isogeny class;
 \item $5$ are not invertible in their multiplicator ring;
 \item $8$ classes admit principal polarizations;
 \item $10$ isomorphism classes of princ. polarized AV.
\end{itemize}
\end{frame}
\begin{frame}{Example}
Concretely:
{\scriptsize \begin{align*}
  \begin{split} 
  I_1 = & 2645633792595191 \Z \oplus (F + 836920075614551) \Z \oplus (F^2 + 1474295643839839)\Z \oplus\\
	& \oplus (F^3 + 1372829830503387)\Z \oplus (F^4 + 1072904687510)\Z \oplus\\
	& \oplus \frac{1}{3}(F^5 + F^4 + F^3 + 2F^2 + 2F + 6704806986143610)\Z \oplus\\
	& \oplus \frac{1}{9}(F^6 + F^5 + F^4 + 8F^3 + 2F^2 + 2991665243621169) \Z \oplus\\
	& \oplus \frac{1}{27}(F^7 + F^6 + F^5 + 17F^4 + 20F^3 + 9F^2 + 68015312518722201)\Z\\
  \end{split}
\intertext{principal polarizations:}
  \begin{split}
  x_{1,1} = \frac{1}{27}( & -121922F^7 + 588604F^6 - 1422437F^5 +\\
			  & +1464239F^4 + 1196576F^3 - 7570722F^2 + 15316479F - 12821193)\\ 
%   \end{split}\\
%   \begin{split}
  x_{1,2} = \frac{1}{27}( & 3015467F^7 - 17689816F^6 + 35965592F^5 -\\
			  & -64660346F^4 + 121230619F^3 - 191117052F^2 + 315021546F - 300025458)\\
  \end{split}\\
  & \End(I_1) =  R\\
  & \#\Aut(I_1,x_{1,1}) = \#\Aut(I_1,x_{1,2}) = 2
 \end{align*}}
\end{frame}


\begin{frame}{Example}
 
{\scriptsize \begin{align*}
  \begin{split} 
  I_7 = & 2\Z\oplus(F + 1)\Z\oplus(F^2 + 1)\Z\oplus(F^3 + 1)\Z\oplus(F^4 + 1)\Z\oplus\frac13(F^5 + F^4 + F^3 + 2F^2 + 2F + 3)\Z \oplus \\ 		      & \oplus\frac{1}{36}(F^6 + F^5 + 10F^4 + 26F^3 + 2F^2 + 27F + 45)\Z\oplus\\
	& \oplus \frac{1}{216}(F^7 + 4F^6 + 49F^5 + 200F^4 + 116F^3 + 105F^2 + 198F + 351)\Z\\
  \end{split}
\intertext{principal polarization:}\\[-7ex]
  \begin{split}
  x_{7,1} = \frac{1}{54}(20F^7 - 43F^6 + 155F^5 - 308F^4 + 580F^3 - 1116F^2 + 2205F - 1809)
  \end{split}\\
  \begin{split}
  \End(I_7) & = \Z \oplus  F\Z \oplus  F^2\Z \oplus  F^3\Z \oplus  F^4\Z \oplus
  \frac{1}{3}(F^5 + F^4 + F^3 + 2F^2 + 2F)\Z \oplus \\
	& \oplus \frac{1}{18}(F^6 + F^5 + 10F^4 + 8F^3 + 2F^2 + 9F + 9)\Z \oplus\\
	& \oplus \frac{1}{108}(F^7 + 4F^6 + 13F^5 + 56F^4 + 80F^3 + 33F^2 + 18F + 27)\Z\\
  \end{split}\\
  & \#\Aut(I_7,x_{7,1}) = 2
\end{align*}}             
$I_1$ is invertible in $R$, but $I_7$ is not invertible in $\End(I_7)$.
\end{frame}

\begin{frame}{some results from computations}{}
  {\footnotesize
  \begin{tabular}{| c | c | c | c | c | c | c |}
  \hline
		& isogeny cl.     & isom.cl.     & \parbox{3.6 em}{isom.cl.\\no p.pol.} & \parbox{3.7 em}{isom.cl.\\w/p.pol.} & \parbox{3.3 em}{isom.w/\\${\End=\cO_K}$} & \parbox{3.6 em}{isom.cl.\\no p.pol.\\${\End=\cO_K}$}\\\hline
     $\F_2,g=2$ & $14/34$         & $21$	  & $7$ 	       & $15$	& $15$	& $3$ \\\hline     
     $\F_3,g=2$ & $36/62$         & $76$	  & $23$ 	       & $59$	& $43$	& $6$\\\hline
     $\F_5,g=2$ & $94/128$        & $457$	  & $203$ 	       & $290$	& $159$	& $34$\\\hline
     $\F_7,g=2$ & $168/207$       & $1324$  	  & $636$ 	       & $797$	& $387$	& $88$\\\hline
     $\F_{11},g=2$ & $352/400$    & $4925$  	  & $2675$ 	       & $2797$	& $1476$& $459$\\\hline
     $\F_2,g=3$ & $82/210$        & $226$	  & $102$ 	       & $142$	& $112$	& $16$\\\hline
\red{$\F_3,g=3$}& $390/670$	  & $2564$  	  & $1292$	       & $1548$	& $922$	& $190$\\\hline     
\red{$\F_5,g=3$}& $2274/2994$	  & $65500$	  & $40094$ 	       & $32582$& $17588$& $4998$\\\hline     
\red{$\F_7,g=3$}& $325/7968$	  & $35822$  	  & $29063$ 	       & $7723$	& $909$	& $236$\\\hline     
\red{$\F_{11},g=3$}& $259/30530$  & $35974$ 	  & $29027$	       & $8049$	& $965$	& $264$\\\hline
  \end{tabular}
  }\vspace{1 em}

  black = all ordinary squarefree isogeny classes have been computed
  
  \red{red = computation in progress}
\end{frame}

\begin{frame}{ Final remarks }
\begin{itemize}
         \item Using Centeleghe-Stix '15 we can compute the isomorphism classes in $\cC_h$ over $\F_p$  where $h$ is \textbf{square-free} and \textbf{without real roots}.\\
\pause much larger subcategory!!! ... but no polarizations in this case.         
\pause   \item we can also deal with the case $\cC_{h^d}$ (with $h$ square-free) when $\Z[F,q/F]$ is Bass.
\pause   \item base field extensions (ordinary case).
\pause   \item period matrices (ordinary case) of the canonical lift.
% \pause   \item conjugacy classes of integral matrices.
\end{itemize}
\end{frame}

\begin{frame}{ }
\begin{center}
{\Large Thank you!}
\end{center}
\end{frame}

\end{document}
