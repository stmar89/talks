\documentclass[a4paper,12pt]{amsart} %{article}
\usepackage{pacchetticomandi}
\title{The ideal class monoid of an order in a number field}
%\author{Stefano Marseglia, SU}
\date{\today}

%%%%%%%%%%%%%%%%%%%%%%%%%%%%%%%%%%%%%%%%%%%%%%%%%%%%%%%%%%%%%%%%%%%%%%%%%%%%%%%%%%%%%%%%%%%%%%%%%%%%%%%%%%%%%%
\begin{document}
\maketitle
%%%%%%%%%%%%%%%%%%%%%%%%%%%%%%%%%%%%%%%%%%%%%%%%%%%%%%%%%%%%%%%%%%%%%%%%%%%%%%%%%%%%%%%%%%%%%%%%%%%%%%%%%%%%%%%%%%%%%%
Let $R$ be an order in a number field $K$. The \emph{ideal class monoid of $R$} is defined as the set of fractional $R$-ideals modulo $R$-linear isomorphisms, with ideal product as multiplication. If $R$ is the \emph{maximal order} of $K$, then every fractional $R$-ideal is invertible and the ideal class monoid coincides with the class group of $K$. In particular there are well-known algorithms to compute it.
On the other hand, if $R$ is not a Dedekind domain, there are ideals that are not invertible and the situation is more complicated. We describe a method to compute a full set of representatives of the ideal class monoid of $R$ and, if time permits, to describe an application to counting the conjugacy classes of integral matrices with a given irreducible characteristic polynomial.
\end{document}