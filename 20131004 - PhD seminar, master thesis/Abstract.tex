\documentclass[a4paper,12pt]{amsart} %{article}
\usepackage{pacchetticomandi0.4}
\title{Super-multiplicativity of ideal norms in number fields}
%\author{Stefano Marseglia, SU}
\date{\today}

%%%%%%%%%%%%%%%%%%%%%%%%%%%%%%%%%%%%%%%%%%%%%%%%%%%%%%%%%%%%%%%%%%%%%%%%%%%%%%%%%%%%%%%%%%%%%%%%%%%%%%%%%%%%%%
\begin{document}
\maketitle
%%%%%%%%%%%%%%%%%%%%%%%%%%%%%%%%%%%%%%%%%%%%%%%%%%%%%%%%%%%%%%%%%%%%%%%%%%%%%%%%%%%%%%%%%%%%%%%%%%%%%%%%%%%%%%%%%%%%%%

%When we are studying a number ring $R$, that is a subring of a number field $K$, it can be useful to understand ``how big" its ideals are compared to the whole ring. The main tool for this purpose is the norm map:
%\begin{align*}
% N: \cI(R) & \longrightarrow \Z_{> 0} \\
%    I & \longmapsto \# R/I
%\end{align*}
%where $\cI(R)$ is the set of non-zero ideals of $R$. It is well known that this map is multiplicative if $R$ is the \textit{maximal order}, or \textit{ring of integers} of the number field. This means that for every pair of ideals $I,J\subseteq R$ we have:
%\[N(I)N(J)=N(IJ).\]
%For an arbitrary number ring in general this equality fails. For example, if we consider the quadratic order $\Z[2i]$ and the ideal $I=(2,2i)$, then we have that $N(I)=2$ and $N(I^2)=8$, so we have the inequality $N(I^2)> N(I)^2$.
%\\
%
%The inequality of the previous example is not a coincidence. More precisely I proved that in any quadratic order, for every pair of ideals $I,J$ we have that $N(IJ)\geq N(I)N(J)$. We will call the norm of a number ring \textit{super-multiplicative} if this inequality holds for every pair of ideals. A natural question is if the ideal norm is always super-multiplicative. The answer is negative, and it's not difficult to exhibit an example which tells us that in a quartic order we cannot prove an analogous theorem.\\
%
%In a quadratic order every ideal can be generated by 2 elements and in a quartic order by 4 elements, so we are led to wonder if the behaviour of the norm is related to the number of generators and  what happens in a cubic order, or more generally in a number ring in which every ideal can be generated by 3 elements.\\
%
%
%The rest of my master thesis aims to prove the following:
%\begin{thm*}
%Let $R$ be a number ring. Then the following statements are equivalent:
%\begin{enumerate}
%\item \label{impl:1} every ideal of $R$ can be generated by $3$ elements;
%\item \label{impl:2} for every ring extension $R\subseteq  R'\subseteq \tilde{R}$, where $\tilde{R}$ is the normalization of $R$, we have that the norm is super-multiplicative.
%\end{enumerate}
%\end{thm*}
%
%It turns out that this proof of \ref{impl:1} implies \ref{impl:2} holds in a more general setting: if $I,J$ are two ideals in a commutative $1$-dimensional noetherian domain $R$, such that $IJ$ can be generated by 3 elements and the norm $N(IJ)$ is finite then we have $N(IJ)\geq N(I)N(J)$.\\
%
%To the other implication, we will first deal with local number rings, bounding the number of generators of any ideal. Then we will give a sufficient condition on the behaviour of the ideal norm to prove that this bound is $\leq 3$. Finally we will apply this result to the non-local case to complete the proof.

Let $R$ be a commutative ring, $I$ an $R$-ideal. The norm of $I$ is defined as $N(I) = \#\left(R/I\right) = [R:I]$. We say that the norm is super-multiplicative on $R$ (briefly $R$ is S.M.) if for every pair of ideals $I,J$ s.t.\ $[R:IJ]<\infty$ we have $N(IJ)\geq N(I)N(J)$. Observe that if  the norm of an order $R$ satisfies the other inequality for every maximal ideal $\p$, that is $N(\p^2)\leq N(\p)^2$,  then $\dim_{(R/\p)}(\p/\p^2)= 1$, i.e.\ the order is Dedekind and the ideal norm is actually multiplicative. The main goal of my thesis is to give a characterization for a number ring, that is a subfield of a number field, of being S.M.\ in terms of the minimal number of generators of its ideals. I proved:
\begin{thm*}
Let $R$ be a number ring. Then the following statements are equivalent:
\begin{enumerate}
\item \label{impl:1} every ideal of $R$ can be generated by $3$ elements;
\item \label{impl:2} for every ring extension $R\subseteq  R'\subseteq \tilde{R}$, where $\tilde{R}$ is the normalization of $R$, we have that the norm is super-multiplicative.
\end{enumerate}
\end{thm*}




\end{document}