%\documentclass[usenames,dvipsnames]{beamer}
 \documentclass[usenames,dvipsnames,handout]{beamer}

\usetheme{AnnArbor}
% \usecolortheme{default}
% \usecolortheme{crane}
\usecolortheme{beaver}
\usecolortheme{dolphin}
% \usecolortheme{orchid}
% \usecolortheme{rose}


\usepackage{fourier}
\usepackage{faktor}
\usepackage{amssymb}
\usepackage{amsmath}
\usepackage{amsthm}
%\usepackage{stmaryrd}
\usepackage{hyperref}
\usepackage[all]{xy}
\usepackage{tikz}
%    \usetikzlibrary{mindmap,shadows,shapes.geometric,shapes.misc,positioning}
\usepackage{tikz-cd}
\tikzset{
    invisible/.style={opacity=0},
    visible on/.style={alt={#1{}{invisible}}},
    alt/.code args={<#1>#2#3}{%
      \alt<#1>{\pgfkeysalso{#2}}{\pgfkeysalso{#3}}%
  }
}
%\usetikzlibrary{matrix}
%\usetikzlibrary{calc,intersections}
%\newcommand{\downmapsto}{\rotatebox[origin=c]{-90}{$\large\mapsto$}\mkern2mu} %MnSymbol doesn't work well with beamer
\usepackage{multirow}

\def\Q{\mathbb{Q}}
\def\Z{\mathbb{Z}}
\def\C{\mathbb{C}}
\def\R{\mathbb{R}}
\def\F{\mathbb{F}}

\DeclareMathOperator{\AV}{AV}
\DeclareMathOperator{\Mat}{Mat}
\DeclareMathOperator{\Pol}{Pol}
\DeclareMathOperator{\Char}{char}
\DeclareMathOperator{\rk}{Rank}
\DeclareMathOperator{\Frob}{Frob}
\DeclareMathOperator{\ICM}{ICM}
\DeclareMathOperator{\Pic}{Pic}
\DeclareMathOperator{\Aut}{Aut}
\DeclareMathOperator{\Hom}{Hom}
\DeclareMathOperator{\End}{End}
\DeclareMathOperator{\Gal}{Gal}
\DeclareMathOperator{\mSpec}{mSpec}
\DeclareMathOperator{\GL}{GL}
\DeclareMathOperator{\Tr}{Tr}
\DeclareMathOperator{\Jac}{Jac}
%\renewcommand{\char}{char} %CRASHES WITH beamer


\newcommand{\cG}{\mathcal{G}}
%\newcommand{\cB}{{\mathcal B}}
%\newcommand{\cC}{{\mathcal C}}
\newcommand{\cO}{{\mathcal O}}
\newcommand{\cH}{{\mathcal H}}
%\newcommand{\cM}{{\mathcal M}}
\newcommand{\cT}{{\mathcal T}}
%\newcommand{\cW}{{\mathcal W}}


\newcommand{\vphi}{\varphi}

\newcommand{\p}{{\mathfrak p}}
\newcommand{\frf}{{\mathfrak f}}

\newcommand{\set}[1]{\left\lbrace#1\right\rbrace }
\newcommand{\Span}[1]{\left<#1\right>}

%\newcommand{\AVord}[1]{\AV^{\text{ord}}({#1})}
%\newcommand{\Modord}[1]{\cM^{\text{ord}}({#1})}

%\newcommand{\AVcs}[1]{\AV^{\text{cs}}({#1})}
%\newcommand{\Modcs}[1]{\cM^{\text{cs}}({#1})}

\newcommand{\Acan}{\mathcal{A}_{\mathrm{can}}}
\newcommand{\AcanC}{A_{\mathrm{can}}}
\newcommand{\Palpha}[2]{\mathcal{P}^{\alpha}_{{#1}}({#2})}
\newcommand{\Pone}[2]{\mathcal{P}^{1}_{{#1}}({#2})}

\newcommand{\red}[1]{\textcolor{red}{#1}}
\newcommand{\blue}[1]{\textcolor{blue}{#1}}
\newcommand{\green}[1]{\textcolor{ForestGreen}{#1}}

\newtheorem{df}{Definition}[section]
\newtheorem{remark}[df]{Remark}
\newtheorem{prop}[df]{Proposition}
\newtheorem{cor}[df]{Corollary}



%AUTHOR DETAILS
%%%%%%%%%%%%%%%%%%%%%%%%%%%%%%%%%%%%%%%%%%%%%%%%
\title[]{Every finite abelian group is the group of rational points of an
ordinary abelian variety over $\F_2$, $\F_3$ and $\F_5$}
\subtitle{}
\author[Stefano Marseglia]{Stefano Marseglia\\}
\institute[]{Utrecht University}
\date[21 April 2022]{DIAMANT Symposium - 21 April 2022\\ \pause joint work with\\ {\bf Caleb Springer}}

\begin{document}

\begin{frame}
\titlepage
\end{frame}


\begin{frame}{ Abelian Varieties }
    \begin{itemize}
        \item An {\bf abelian variety} over a field $k$ is a projective geometrically connected group variety over $k$.
        \pause
        \item Fun fact: the group of rational points of an AV is commutative...
        \pause
        \item ... but this is not why they are called abelian.
        \pause
        \item e.g. AVs of dim $1$ are elliptic curves:
        \[\text{when }\Char(k)\neq 2,3 \leadsto Y^2=X^3+AX+B  \]
        % \vspace{-2em}
        % \pause
        % \item \red{Goal}: study \textbf{isomorphism classes} of abelian varieties over a \textbf{finite field} (+ extra structure, like polarizations, period matrices, etc.)
        \pause
        \item \vspace{-6mm} Annoying fact: in dimension $g>1$, the equations are typically horrible.
        % \pause
        % \item over $\C$:
        % \[
        %      \set{ \text{abelian varieties $/\C$} } \longleftrightarrow
        %      \set{\parbox[c]{10em}{$\C^g/L$ with $L\simeq \Z^{2g}$ with\\eq.cl.~of Riemann form}}
        % \]
        % \pause \vspace{-6mm} \item in positive characteristic we don't have such equivalence.
	\end{itemize}
\end{frame}

\begin{frame}{ Isogeny classes }
    \begin{itemize}
        \item $A$ and $B$ are {\bf isogenous} if $\dim A=\dim B$ and there exists a surjective homomorphism $\varphi:A\to B$.
        \pause
        \item Being isogenous is an equivalence relation.
        \pause
        \item Let $q$ be a power of a prime $p$.
        \item $A/\F_{q}$ comes with a {\bf Frobenius endomorphism},
        \pause
        that induces an action
        \[ \Frob_A : T_\ell A \rightarrow T_\ell A \text{ for any prime }\ell\neq p, \]
        where $T_\ell A=\varprojlim A[\ell^n](\overline \F_p)$ is the $\ell$-adic Tate module.
        \pause
        \item $T_\ell A\simeq \Z_\ell^{2\dim A}$. Put $h_A=\Char(\Frob_A)$.
        \pause
        \item $h_A$ satisfies:
        \begin{itemize}
            \item the definition of $h_A$ does not depend on $\ell$.
            \item $h_A\in\Z[x]$.
            \pause
            \item $\deg h_A = 2\dim A$
            \item the complex roots of $h_A$ have absolute value $\sqrt{q}$.
        \end{itemize}
    \end{itemize}
\end{frame}

\begin{frame}{ Isogeny classes }
    \begin{itemize}
        \item By {Honda-Tate} theory, the association
        \[ A \mapsto h_A=\Char(\Frob_A)\]
        is {\bf injective up to isogeny}, and...\\
        \pause
        ...allows us to {enumerate} all AVs up to isogeny.
        \pause
        \item Tate: $h_A$ {\bf squarefree} (no multiple roots) iff $\End_{\F_q}(A)$ is commutative.
        \pause
        \item The number of rational points is $\# A(\F_q)=h_A(1)$ isogeny invariant.
        \pause
        \item ..but the group $A(\F_q)$ is not.
        \pause
        \item We say that $A/\F_q$ is {\bf cyclic} if $A(\F_q)$ is cyclic.
    \end{itemize}
\end{frame}

\begin{frame}{ Ordinary abelian varieties }
	\begin{itemize}
        \item $A/\F_q$ is {\bf ordinary} if
         % $A[p](\overline \F_p)\simeq (\Z /p\Z )^{\dim A}$ ...
        % \pause
        % \item
         % ... or equivalently if
         the coefficient of $x^{\dim A}$ in $h_A(x)$ is coprime to $q$.
        \pause
        \item Being ordinary is an isogeny invariant.
	\end{itemize}
    \pause
    \begin{theorem}[Deligne]
        Let $h_A(x)$ be ordinary and squarefree.
        \pause
        Put $K=\Q[F]=\Q[x]/(h_A(x))$.\\
        \pause
        Then we have an {\bf equivalence} of categories:
        \[\begin{array}{c}
        \set{\text{abelian varieties $B/\F_q$ with $h_B(x)=h_A(x)$}}\\
        \updownarrow \\
        \pause
        \set{ \text{fractional ideals of $\Z[F,q/F]\subset K$ } }\\
        \rotatebox{90}{=}\\
        \pause
        \set{{\text{fin.~gen.~$\Z[F,q/F]$-module in $K$, free of max $\mathrm{rank}_\Z$}}}
        \end{array}\]
    \end{theorem}
    \pause Note: $F \leftrightarrow \Frob$ (and $q/F \leftrightarrow \mathrm{Verschiebung}$).
\end{frame}


\begin{frame}{ Cyclic abelian varieties }
    \begin{corollary}
        If $B$ corresponds to the fractional $\Z[F,q/F]$-ideal $J$ then
        \pause
        \[ B(\F_q) \simeq \frac{J}{(1-F)J}.\]
    \end{corollary}
    \pause
    \begin{prop}[M.-Springer]
        Every ordinary squarefree isogeny class contains a cyclic abelian variety.
    \end{prop}
    \pause
	Proof: Take $B\longleftrightarrow J=\Z[F,q/F]$.
    \pause
    \[ B(\F_q)\simeq \frac{\Z[F,q/F]}{(1-F)} \simeq \frac{\Z[x,y]}{(h_A(x),xy-q,x-1)}
    \pause
    \simeq \frac{\Z}{h_A(1)\Z}.\qed \]
\end{frame}

\begin{frame}{ Number of points }
    \begin{theorem}[Howe-Kedlaya]
        Let $m\in\Z_{\geq 0}$. Then there is a squarefree ordinary $A/\F_2$ such that $\#A(\F_2)=m$.
    \end{theorem}
    \pause
    \begin{theorem}[van Bommel-Costa-Li-Poonen-Smith]
        Let $m\in\Z_{\geq 0}$ and $k$ be $\F_3$ or $\F_5$. Then there is a squarefree ordinary $A/k$ such that $\#A(k)=m$.
    \end{theorem}
    \pause
    They use extremely clever constructions that allows them to construct characteristic polynomials $h_A$ such that $h_A(1)=m$.
\end{frame}

\begin{frame}{ Group of points }
    \begin{theorem}[M.-Springer]
        Let $k$ be $\F_2$, $\F_3$ or $\F_5$. Let $G$ be an abelian group.
        \pause
        Then there exists an ordinary $A/k$ with $A(k) = G$.
    \end{theorem}
    \pause
    Proof: Write
    \[ G\simeq \frac{\Z}{m_1 \Z} \times \ldots \times \frac{\Z}{m_s \Z}.\]
    \pause
    By H-K or vBCLPS, for each $i$ there is an isogeny class with $m_i$ points.\\
    \pause
    By Proposition, within each of the isogeny classes, there is a cyclic $A_i$.\\
    \pause
    Take $A=\prod_i A_i$. \qed
    \pause
    \begin{corollary}
        If $G$ is cyclic we can take $A$ to be ordinary and squarefree.
    \end{corollary}
\end{frame}

\begin{frame}{ Further results (building on vBCLPS) }
	\begin{itemize}
        \pause
        \item Over $\F_4$: for every abelian $G\neq 0$ there exists an ordinary or almost ordinary $A/\F_4$ such that $A(\F_4)\simeq G$.
        \pause
        \item Over $\F_7$: for every cyclic $G\neq 0$ with $\# G\not \in \set{2,8,14,16,17,73}$ there exists a squarefree ordinary $A/\F_7$ such that $A(\F_7)\simeq G$.
        \pause
        \item vBCLPS: For an arbitrary $q$, every integer $m\geq q^{3\sqrt{q}\log q}$ arises as $m=\#A(\F_q)$ for some ordinary squarefree $A/\F_q$.
	\end{itemize}
    \pause
    \begin{theorem}[M.-Springer]
        Let $m_1,\ldots,m_r$ be integers satisfying $m_i\geq q^{3\sqrt{q}\log q}$.
        Put
        \[ G=\frac{\Z}{m_1 \Z}\times \ldots \times \frac{\Z}{m_r \Z}. \]
        \pause
        Then there is an ordinary $A/\F_q$ such that $G=A(\F_q)$.
    \end{theorem}
\end{frame}

% \begin{frame}{  }
% 	\begin{itemize}
%         \item
%         \item
% 	\end{itemize}
% \end{frame}
%
% \begin{frame}{  }
% 	\begin{itemize}
%         \item
%         \item
% 	\end{itemize}
% \end{frame}

% \begin{frame}{  }
% 	\begin{itemize}
%         \item
%         \item
% 	\end{itemize}
% \end{frame}

\begin{frame}{  }
    \begin{center}
    \green{\huge Thank you!}
    \end{center}
\end{frame}

\end{document}
