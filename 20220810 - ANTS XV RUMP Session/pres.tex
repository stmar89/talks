\documentclass[usenames,dvipsnames]{beamer}
%  \documentclass[usenames,dvipsnames,handout]{beamer}

\usetheme{AnnArbor}
% \usecolortheme{default}
% \usecolortheme{crane}
\usecolortheme{beaver}
\usecolortheme{dolphin}
% \usecolortheme{orchid}
% \usecolortheme{rose}


\usepackage{fourier}
\usepackage{faktor}
\usepackage{amssymb}
\usepackage{amsmath}
\usepackage{amsthm}
\usepackage{faktor}
%\usepackage{stmaryrd}
\usepackage{hyperref}
\usepackage[all]{xy}
\usepackage{tikz}
%    \usetikzlibrary{mindmap,shadows,shapes.geometric,shapes.misc,positioning}
\usepackage{tikz-cd}
% \tikzset{
%     invisible/.style={opacity=0},
%     visible on/.style={alt={#1{}{invisible}}},
%     alt/.code args={<#1>#2#3}{%
%       \alt<#1>{\pgfkeysalso{#2}}{\pgfkeysalso{#3}}%
%   }
% }
%\usetikzlibrary{matrix}
%\usetikzlibrary{calc,intersections}
%\newcommand{\downmapsto}{\rotatebox[origin=c]{-90}{$\large\mapsto$}\mkern2mu} %MnSymbol doesn't work well with beamer
\usepackage{multirow}

\def\Q{\mathbb{Q}}
\def\Z{\mathbb{Z}}
\def\C{\mathbb{C}}
\def\R{\mathbb{R}}
\def\F{\mathbb{F}}

\DeclareMathOperator{\AV}{AV}
\DeclareMathOperator{\Mat}{Mat}
\DeclareMathOperator{\Pol}{Pol}
\DeclareMathOperator{\Char}{char}
\DeclareMathOperator{\rk}{Rank}
\DeclareMathOperator{\Frob}{Frob}
\DeclareMathOperator{\ICM}{ICM}
\DeclareMathOperator{\Pic}{Pic}
\DeclareMathOperator{\Aut}{Aut}
\DeclareMathOperator{\Hom}{Hom}
\DeclareMathOperator{\End}{End}
\DeclareMathOperator{\Gal}{Gal}
\DeclareMathOperator{\mSpec}{mSpec}
\DeclareMathOperator{\GL}{GL}
\DeclareMathOperator{\Tr}{Tr}
\DeclareMathOperator{\Jac}{Jac}
%\renewcommand{\char}{char} %CRASHES WITH beamer


\newcommand{\cG}{\mathcal{G}}
%\newcommand{\cB}{{\mathcal B}}
%\newcommand{\cC}{{\mathcal C}}
\newcommand{\cO}{{\mathcal O}}
\newcommand{\cH}{{\mathcal H}}
%\newcommand{\cM}{{\mathcal M}}
\newcommand{\cT}{{\mathcal T}}
%\newcommand{\cW}{{\mathcal W}}


\newcommand{\vphi}{\varphi}

\newcommand{\p}{{\mathfrak p}}
\newcommand{\frf}{{\mathfrak f}}

\newcommand{\set}[1]{\left\lbrace#1\right\rbrace }
\newcommand{\Span}[1]{\left<#1\right>}

%\newcommand{\AVord}[1]{\AV^{\text{ord}}({#1})}
%\newcommand{\Modord}[1]{\cM^{\text{ord}}({#1})}

%\newcommand{\AVcs}[1]{\AV^{\text{cs}}({#1})}
%\newcommand{\Modcs}[1]{\cM^{\text{cs}}({#1})}

\newcommand{\Acan}{\mathcal{A}_{\mathrm{can}}}
\newcommand{\AcanC}{A_{\mathrm{can}}}
\newcommand{\Palpha}[2]{\mathcal{P}^{\alpha}_{{#1}}({#2})}
\newcommand{\Pone}[2]{\mathcal{P}^{1}_{{#1}}({#2})}

\newcommand{\red}[1]{\textcolor{red}{#1}}
\newcommand{\blue}[1]{\textcolor{blue}{#1}}
\newcommand{\green}[1]{\textcolor{ForestGreen}{#1}}

\newtheorem{df}{Definition}[section]
\newtheorem{remark}[df]{Remark}
\newtheorem{prop}[df]{Proposition}
\newtheorem{cor}[df]{Corollary}



%AUTHOR DETAILS
%%%%%%%%%%%%%%%%%%%%%%%%%%%%%%%%%%%%%%%%%%%%%%%%
\title[RUMP Session ANTS XV]{Welcome to your Linear Algebra 1 exam!}
\subtitle{}
\author[Stefano Marseglia]{}
% \institute[]{}
\date[10 August 2022]{}

\begin{document}
\begin{frame}{}
   \maketitle
   \begin{center}
      \pause Don't forget to motivate your answers.\\
      \pause The use of the (Magma) calculator is allowed.
   \end{center}
\end{frame}


\begin{frame}{}
   Recall:\\
   $A,B \in \Mat_{n\times n}$ are {\bf conjugate} ($A\sim B$) if $AP=PB$ for some $P\in \GL_n$.\\
   {\bf Question 1:} 
   \pause Are the following two matrices conjugate
   \[
   A=\begin{pmatrix}
      0 & -1 \\ 5 & 0
   \end{pmatrix}, \quad
   B=\begin{pmatrix}
      -1 & 2 \\ -3 & 1
   \end{pmatrix}?
   \]
   \pause
   {\bf Answer(s):}\\ 
   \pause Over $\Q$: yes! Same minimal and characteristic polynomial $x^2+5$.\\
   \pause {\bf But...}\\    
   \pause Over $\Z$: no! Every such a $P$ must have even determinant.
\end{frame}

\begin{frame}{}
   Fix minimal polynomial $m=m_1\cdots m_n$ (squarefree) and characteristic polynomial $h=m_1^{s_1}\cdots m_n^{s_n}$:\\
   \pause {\bf Question 2.1}
   Find the representatives of the $\Z$-conjugacy classes of matrices with min.~polynomial $m$ and characteristic polynomial $h$.\\
   \pause
   {\bf Answer:}
   \begin{theorem}[(generalized) Latimer-MacDuffee]
      \[ \begin{array}{cc}
         \faktor{\set{\text{matrices with min.~poly.~$m$ and char.~poly.~$h$}}}{\sim_\Z}\\
         \pause \updownarrow\\
         \faktor{\set{\parbox[p]{19em}{$\Z$-lattices in 
            $V= \left(\frac{\Q[x]}{m_1}\right)^{s_1}
            \times \ldots \times 
            \left(\frac{\Q[x]}{m_n}\right)^{s_n}$
            closed under multiplication by $\pi:=x\bmod m$
            }}}{\simeq_{\Z[\pi]}}
      \end{array} \]
   \end{theorem}
   \pause {\bf Question 2.2} How do you {\bf compute} these $\Z[\pi]$-modules?\\
   \pause
   {\bf Answer:} ....check the arXiv in the next couple of days....
\end{frame}


\begin{frame}{}\
   {\bf Bonus Question for extra points:}\\
   \pause How do you compute abelian varieties over $\F_q$ with ordinary characteristic polynomial of Forbenius $h=m_1^{s_1}\cdots m_n^{s_n}$ (up to $\F_q$-isomorphism)?\\
   \pause
   {\bf Answer:} Do the same thing with $\Z[\pi,q/\pi]$ instead of $\Z[\pi]$:
   \pause 
   \begin{theorem}[Thank you Deligne]
      \[ \begin{array}{cc}
         \faktor{\set{\text{abelian varieties with char.~poly. $h$}}}{\simeq_{\F_q}}\\
         \pause \updownarrow\\
         \faktor{\set{\parbox[p]{19em}{$\Z$-lattices in 
            $V= \left(\frac{\Q[x]}{m_1}\right)^{s_1}
            \times \ldots \times 
            \left(\frac{\Q[x]}{m_n}\right)^{s_n}$
            closed under multiplication by $\pi:=x\bmod m$ and $q/\pi$
            }}}{\simeq_{\Z[\pi,q/\pi]}}
      \end{array} \]
   \end{theorem}
\end{frame}

\begin{frame}{}\
   \begin{center}
      {\Huge Congrats: you passed the exam!}
   \end{center}
\end{frame}

\end{document}
