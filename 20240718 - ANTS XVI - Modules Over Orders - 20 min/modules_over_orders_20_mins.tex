\documentclass[usenames,dvipsnames]{beamer}
%  \documentclass[usenames,dvipsnames,handout]{beamer}

\usetheme{AnnArbor}
% \usecolortheme{default}
% \usecolortheme{crane}
\usecolortheme{beaver}
\usecolortheme{dolphin}
% \usecolortheme{orchid}
% \usecolortheme{rose}


\usepackage{fourier}
\usepackage{faktor}
\usepackage{amssymb}
\usepackage{amsmath}
\usepackage{amsthm}
\usepackage{faktor}
%\usepackage{stmaryrd}
\usepackage{hyperref}
\usepackage[all]{xy}
\usepackage{tikz}
%    \usetikzlibrary{mindmap,shadows,shapes.geometric,shapes.misc,positioning}
\usepackage{tikz-cd}
% \tikzset{
%     invisible/.style={opacity=0},
%     visible on/.style={alt={#1{}{invisible}}},
%     alt/.code args={<#1>#2#3}{%
%       \alt<#1>{\pgfkeysalso{#2}}{\pgfkeysalso{#3}}%
%   }
% }
%\usetikzlibrary{matrix}
%\usetikzlibrary{calc,intersections}
%\newcommand{\downmapsto}{\rotatebox[origin=c]{-90}{$\large\mapsto$}\mkern2mu} %MnSymbol doesn't work well with beamer
\usepackage{multirow}

\def\Q{\mathbb{Q}}
\def\Z{\mathbb{Z}}
\def\C{\mathbb{C}}
\def\R{\mathbb{R}}
\def\F{\mathbb{F}}

\DeclareMathOperator{\AV}{AV}
\DeclareMathOperator{\Mat}{Mat}
\DeclareMathOperator{\Pol}{Pol}
\DeclareMathOperator{\Char}{char}
\DeclareMathOperator{\rk}{Rank}
\DeclareMathOperator{\Frob}{Frob}
\DeclareMathOperator{\ICM}{ICM}
\DeclareMathOperator{\Pic}{Pic}
\DeclareMathOperator{\Aut}{Aut}
\DeclareMathOperator{\Hom}{Hom}
\DeclareMathOperator{\End}{End}
\DeclareMathOperator{\Gal}{Gal}
\DeclareMathOperator{\mSpec}{mSpec}
\DeclareMathOperator{\GL}{GL}
\DeclareMathOperator{\Tr}{Tr}
\DeclareMathOperator{\Jac}{Jac}
%\renewcommand{\char}{char} %CRASHES WITH beamer


\newcommand{\cG}{\mathcal{G}}
%\newcommand{\cB}{{\mathcal B}}
%\newcommand{\cC}{{\mathcal C}}
\newcommand{\cO}{{\mathcal O}}
\newcommand{\cH}{{\mathcal H}}
\newcommand{\cL}{{\mathcal L}}
%\newcommand{\cM}{{\mathcal M}}
\newcommand{\cT}{{\mathcal T}}
%\newcommand{\cW}{{\mathcal W}}


\newcommand{\vphi}{\varphi}

\newcommand{\p}{{\mathfrak p}}
\newcommand{\frf}{{\mathfrak f}}

\newcommand{\set}[1]{\left\lbrace#1\right\rbrace }
\newcommand{\Span}[1]{\left<#1\right>}

%\newcommand{\AVord}[1]{\AV^{\text{ord}}({#1})}
%\newcommand{\Modord}[1]{\cM^{\text{ord}}({#1})}

%\newcommand{\AVcs}[1]{\AV^{\text{cs}}({#1})}
%\newcommand{\Modcs}[1]{\cM^{\text{cs}}({#1})}

\newcommand{\Acan}{\mathcal{A}_{\mathrm{can}}}
\newcommand{\AcanC}{A_{\mathrm{can}}}
\newcommand{\Palpha}[2]{\mathcal{P}^{\alpha}_{{#1}}({#2})}
\newcommand{\Pone}[2]{\mathcal{P}^{1}_{{#1}}({#2})}

\newcommand{\red}[1]{\textcolor{red}{#1}}
\newcommand{\blue}[1]{\textcolor{blue}{#1}}
\newcommand{\green}[1]{\textcolor{ForestGreen}{#1}}

\newtheorem{df}{Definition}[section]
\newtheorem{remark}[df]{Remark}
\newtheorem{prop}[df]{Proposition}
\newtheorem{cor}[df]{Corollary}



%AUTHOR DETAILS
%%%%%%%%%%%%%%%%%%%%%%%%%%%%%%%%%%%%%%%%%%%%%%%%
\title[ANTS XVI - MIT]{Modules over orders,\\conjugacy classes of integral matrices and\\ abelian varieties over finite fields}
\subtitle{}
\author[Stefano Marseglia]{}
% \institute[]{}
\date[July 18 2024]{}

\begin{document}
\begin{frame}{}
   \maketitle
   % \begin{center}
   %    \pause Don't forget to motivate your answers.\\
   %    \pause The use of the (Magma) calculator is allowed.
   % \end{center}
\end{frame}


\begin{frame}{}
   \begin{itemize}
      \item Let $R$ be a commutative ring with unity.
      \item $A,B \in \Mat_{n\times n}(R)$ are {\bf $R$-conjugate} ($A\sim_R B$) if $AP=PB$ for some $P\in \GL_n(R)$.
      \item The minimal polynomial of $A \in \Mat_{n\times n}(R)$ is the polynomial of smallest degree such that $m(A) = O$ (the zero $n\times n$ matrix).
      \item The characteristic polynomial of $A \in \Mat_{n\times n}(R)$ is $\det(A-xI_n)$.
   \end{itemize}
   {\bf Question 1:} 
   \pause Are the following two matrices $\Q$-conjugate? Are they $\Z$-conjugate?
   \[
   A=\begin{pmatrix}
      0 & -1 \\ 5 & 0
   \end{pmatrix}, \quad
   B=\begin{pmatrix}
      -1 & 2 \\ -3 & 1
   \end{pmatrix}
   \]
   \pause
   {\bf Answer(s):}\\ 
   \pause Over $\Q$: yes! Same characteristic polynomial $x^2+5$, which is irreducible.\\
   \pause {\bf But...}\\    
   \pause Over $\Z$: no! Every such a $P$ must have even determinant.
\end{frame}

\begin{frame}{}
   {\bf Set-up} 
   Fix monic polynomials $m=m_1\cdots m_n$ and $h=m_1^{s_1}\cdots m_n^{s_n}$ in $\Z[x]$ with 
   \begin{itemize}
      \item each $m_i$ irreducible and 
      \item $m_i\neq m_j$ if $i\neq j$. (i.e.~$m$ is squarefree)
   \end{itemize}
   
   \pause {\bf Question 2.1}
   Can we describe the representatives of the $\Z$-conjugacy classes of matrices with:
   \begin{itemize}
      \item minimal polynomial $m$, and
      \item characteristic polynomial $h$?
   \end{itemize}
   \pause
   {\bf Answer:}
   \begin{theorem}[(generalized) Latimer-MacDuffee]
      The order $\Z[\pi]=\frac{\Z[x]}{(m)}$ acts on $V=\left(\frac{\Q[x]}{m_1}\right)^{s_1}
      \times \ldots \times 
      \left(\frac{\Q[x]}{m_n}\right)^{s_n}$.\\
      We have a bijection
      \[ \begin{array}{cc}
         \faktor{\set{\parbox[p]{7.5em}{$\Z[\pi]$-lattices in $V$}}}
         {\simeq_{\Z[\pi]}}\\
         \pause \updownarrow\\
         \faktor{\set{\text{matrices with min.~poly.~$m$ and char.~poly.~$h$}}}{\sim_\Z}\\
      \end{array} \]
   \end{theorem}
\end{frame}

\begin{frame}
   Proof (idea):\\

\end{frame}

\begin{frame}{}\
   \noindent{\bf Question 3} 
   How do you compute abelian varieties over $\F_q$ with ordinary characteristic polynomial of Frobenius $h=m_1^{s_1}\cdots m_n^{s_n}$ (up to $\F_q$-isomorphism)?\\
   \pause
   {\bf Answer:} Do the same thing with $\Z[\pi,q/\pi]$ instead of $\Z[\pi]$:
   \pause 
   \begin{theorem}[Deligne]
      \[ \begin{array}{cc}
         \faktor{\set{\text{abelian varieties with char.~poly. $h$}}}{\simeq_{\F_q}}\\
         \pause \updownarrow\\
         \faktor{\set{\parbox[p]{19em}{$\Z$-lattices in 
            $V= \left(\frac{\Q[x]}{m_1}\right)^{s_1}
            \times \ldots \times 
            \left(\frac{\Q[x]}{m_n}\right)^{s_n}$
            closed under multiplication by $\pi:=x\bmod m$ and $q/\pi$
            }}}{\simeq_{\Z[\pi,q/\pi]}}
      \end{array} \]
   \end{theorem}
\end{frame}

\begin{frame}
   How do we make this theorems effective?
\end{frame}

\begin{frame}{}\
   {\bf Set-up}:
   \begin{itemize}
      \item $K_1,\ldots,K_n$ number fields, with ring of integers $\cO_i\subset K_i$.
      \item $K=K_1\times \ldots \times K_n$.
      \item $\cO=\cO_1\times \ldots \times \cO_n$, the maximal order of $K$.
      \item $s_1,\ldots,s_n$ positive integers and $V = K_1^{s_1}\times \ldots\times K_n^{s_n}$.
      \item for an order $R$ in $K$, set $\cL(R,V) = \set{\text{$R$-lattice in $V$}}/\simeq_R$.
   \end{itemize}
   {\bf Proposition (Steinitz)}:
   Let~$M$ be in~$\cL(\cO,V)$.
   Then there are fractional~$\cO_i$-ideals~$I_i$ and there exists an~$\cO$-linear isomorphism
   \[ M\simeq
   \bigoplus_{i=1}^n \left(\cO_i^{\oplus(s_i-1)}\oplus I_i\right).
   \]
   The isomorphism class of~$M$ is uniquely determined by 
   % the integers~$s_i$ and 
   the isomorphism class of the fractional~$\cO$-ideal~$I=I_1\oplus \cdots \oplus I_n$.
\end{frame}

\begin{frame}{}\
   \begin{itemize}
      \item Let $\frf=(R:\cO)=\set{x : x \in K x\cO \subseteq R}$ be the conductor of $R$ in $\cO$.
      \item Write $\frf=\oplus_{i=1}^n\frf_i$, $\frf_i$ a fractional $\cO_i$-ideal in $K_i$.
   \end{itemize}
   {\bf Theorem}:
   Let~$M$ be in~$\cL(R,V)$.
   Then there exist an~$M'$ in~$\cL(R,V)$, and fractional~$\cO_i$-ideals~$I_i$ such that
   \begin{itemize}
      \item~$M'\simeq M$ as an~$R$-module.
      \item~$M'\cO = \bigoplus_{i=1}^n \left(\cO_i^{\oplus(s_i-1)}\oplus I_i\right)$.
      \item~$\bigoplus_{i=1}^n \left(\frf_i^{\oplus(s_i-1)}\oplus \frf_iI_i\right) \subseteq M' \subseteq
      \bigoplus_{i=1}^n \left(\cO_i^{\oplus(s_i-1)}\oplus I_i\right)$.
   \end{itemize}
   {\bf Proof}:
\end{frame}

\begin{frame}{}\
   IsIsomorphic
\end{frame}

\begin{frame}{}\
   the algorithm
\end{frame}


\begin{frame}{}\
   reduced the number of enumerations to $1$
\end{frame}

% \begin{frame}{}\
%    \begin{center}
%       {\Huge Congrats: you passed the exam!}
%    \end{center}
% \end{frame}

\end{document}
