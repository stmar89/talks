% \documentclass{beamer}
\documentclass[handout]{beamer}


% \usetheme{CambridgeUS}
\usetheme{Pittsburgh}
\usepackage{commands_beamer}
\usepackage{faktor}
\usepackage{xfrac} 

%AUTHOR DETAILS
%%%%%%%%%%%%%%%%%%%%%%%%%%%%%%%%%%%%%%%%%%%%%%%%
\title[]{Computing isomorphism classes of abelian varieties over finite fields}
\author[Marseglia Stefano]{Marseglia Stefano}
\institute[]{Stockholms Universitet}
\date{21 June 2017}

\begin{document}

\begin{frame}
\titlepage
\end{frame}

\begin{frame}{ Introduction }
\begin{itemize}
 \item Goal: compute \textbf{isomorphism classes} of (principally polarized) abelian varieties over a finite field.
 \pause \item We start from the \textbf{isogeny} classification (\textbf{Honda-Tate}):
 pick $A/\F_q$ and let $h_A(x)$ be the characteristic polynomial of the $\Frob_A$ acting on $T_lA$. We have
 \[A\sim_{\F_q} B_1^{n_1}\cdots B_r^{n_r},\]
 where the $B_i$'s are simple and pairwise non-isogenous, and 
 \[h_A(x) = h_{B_1}(x)^{n_1}\cdots h_{B_r}(x)^{n_r},\]
 where the $h_{B_i}(x)$'s are (specific) powers of irreducible $q$-Weil polynomials.
\end{itemize}
\end{frame}

\begin{frame}{ Deligne's equivalence }
\begin{theorem}[Deligne '69]
Let $q=p^r$, with $p$ a prime. There is an equivalence of categories:
\[\begin{array}{cc}
\set{\text{\textbf{Ordinary} abelian varieties over $\F_q$}}	& A \\
\updownarrow							& \downmapsto \\
\set{\parbox[p]{19em}{pairs $(T,F)$, where $T\simeq_\Z \Z^{2g}$ and $T\overset{F}{\to} T$ s.t.\\
- $F\otimes \Q$ is semisimple\\
- the roots of $\Char_{F\otimes\Q}(x)$ have abs. value $\sqrt{q}$\\
- \textbf{half of them are $p$-adic units}\\
- $\exists V:T\to T$ such that $FV=VF=q$
}}	& (T(A),F(A))
\end{array}\]
\end{theorem}
\begin{remark}
\begin{itemize}
 \item If $\dim(A)=g$ then $\rk(T(A))=2g$;
 \item $\Frob(A)\rightsquigarrow F(A)$.
\end{itemize}
\end{remark}
\end{frame}

\begin{frame}{ Deligne's equivalence }
Fix a \textbf{square-free} characteristic $q$-Weil polynomial $h$.\\
Let $\cC_h$ be the corresponding isogeny class.\\
Denote with $K$ the \'etale algebra $\Q[x]/(h)$ and put $F:= x \mod h$.

\pause Deligne's equivalence induces:

\[\begin{array}{cc}
\faktor{\set{\text{Ordinary abelian varieties over $\F_q$ in $\cC_h$}}}{\simeq} & \\
\updownarrow & \\
\faktor{\set{ \text{fractional ideals of $\Z[F,q/F] \subset K$ } }}{\simeq} & =:\textcolor{blue}{\ICM(\Z[F,q/F])}) 
\end{array}\]
\end{frame}

\begin{frame}{ Centeleghe/Stix's equivalence }
\begin{theorem}[Centeleghe/Stix 2015]
There is an equivalence of categories:
\[\begin{array}{c}
\set{\parbox{17.5em}{Abelian varieties over $\F_p$ such that $\sqrt{p}$ \textbf{does not belong} to their Weil support}}\\
\updownarrow\\
\set{\parbox[p]{19em}{pairs $(T,F)$, where $T\simeq_\Z \Z^{2g}$ and $T\overset{F}{\to} T$ s.t.\\
- $F\otimes \Q$ is semisimple\\
- the roots of $\Char_{F\otimes\Q}(x)$ have abs. value $\sqrt{p}$\\
- \textbf{$\sqrt{p}$ is not a root of $\Char_{F\otimes\Q}(x)$}\\
- $\exists V:T\to T$ such that $FV=VF=p$
}}
\end{array}\]
\end{theorem}

\pause For a $p$-Weil \textbf{square-free} characteristic polynomial $h$ with $h(\sqrt{p})\neq 0$:
\[\faktor{\set{ \text{Abelian varieties in $\cC_h$} }}{\simeq} \longleftrightarrow \textcolor{blue}{\ICM(\Z[F,p/F])}  \]

\end{frame}

\begin{frame}{ICM : Ideal Class Monoid}
Let $R$ be an order in a \'etale  $\Q$-algebra $K$ and $\cO_K$ the ring of integers of $K$.\\
% according to Bourbaki etale (over a field K) implies commutative and finite (since it is defined as being isomorphic to L^n for some extension L of K).
Recall: for fractional $R$-ideals $I$ and $J$
\[ I\simeq_R J \Longleftrightarrow \exists x \in K^\times \text{ s.t.~} xI=J \]
Define 
\[\ICM(R) := \faktor{\set{\text{fractional $R$-ideals}}}{\simeq_R}\]
\pause \begin{itemize}
 \item $\ICM(R)$ is a finite monoid: use the Minkowski bound: SLOW!
% \end{itemize}

% \begin{prop}
% \vspace{-5mm}
\item\[ \ICM(R) \supseteq \bigsqcup_{R\subseteq S \subseteq \cO_K} \Pic(S).\]
% \end{prop}
\end{itemize}
\end{frame}

\begin{frame}{ Weak equivalence }
\begin{theorem}[Dade, Taussky, Zassenhaus '62]
 Two fractional $R$-ideals $I$ and $J$ are \textbf{weakly equivalent} ($I\sim_{\text{wk}} J$) if one of the following equivalent conditions hold:
 \begin{itemize}
  \item $I_\p\simeq_{R_\p} J_\p$ for every $\p\in \mSpec(R)$;
  \item $1\in (I:J)(J:I)$;
  \item $(I:I)=(J:J)$ and $\exists$ an invertible $(I:I)$-ideal $L$ s.t.~$I=LJ$.
 \end{itemize}
\end{theorem}

\pause Notation: for any order $R$:
\begin{itemize}
 \item $\cW(R):= \faktor{\set{\text{fractional $R$-ideals}}}{\sim_{\text{wk}}} $;
 \item $\overline\cW(R):= \faktor{\set{\text{fractional $R$-ideals $I$ with $(I:I)=R$}}}{\sim_{\text{wk}}} $;
 \item $\overline\ICM(R):= \faktor{\set{\text{fractional $R$-ideals $I$ with $(I:I)=R$}}}{\simeq_R} $
\end{itemize}
\end{frame}

\begin{frame}{ Compute $\cW(R)$ and $\ICM(R)$ }
Let $\frf_R=(R:\cO_K)$ be the conductor of $R$ and $I$ a fractional $R$-ideal.\\
Without changing the weak eq.~class, we can assume that
\[I\cO_K=\cO_K.\]
Hence $\frf_R\subseteq I \subseteq \cO_K$, and therefore:
\[\begin{array}{cc}
&\cW(R) \overset{\sim_{\text{wk}}}{\twoheadleftarrow} \set{\text{ fractional $R$-ideals $I$ : $I\cO_K = \cO_K$ }} \\
& \rotatebox{-90}{$\subseteq$} \\
& \set{\text{sub-$R$-modules of $\faktor{\cO_K}{\frf_R}$}} 
\end{array}\]
\pause \begin{theorem}
The action of $\Pic(R)$ on $\overline\ICM(R)$ is free and the quotient is precisely $\overline\cW(R)$.
In particular, we can compute:
\[\ICM(R) = \bigsqcup_{R\subseteq S \subseteq \cO_K} \overline \ICM(S).\]
\end{theorem}
\end{frame}

\begin{frame}{ Dual variety/Polarization }
\begin{itemize}
 \item Howe defined a notion of \textbf{dual} module and of \textbf{polarization} in the category of Deligne modules (\textbf{ordinary} case).
%  \item Fix an $\overline \Q_p \overset{\epsilon}{\simeq} \C$ and choose a \textbf{CM-type} as follows:
%  \[\Phi:=\set{ \vphi:\Z[F,V]\otimes \Q \to \C : v_{p,\epsilon}(\vphi(F))>0 }\]
\pause \item Concretely, if $A\leftrightarrow I$, then $A^\vee \leftrightarrow \overline{I}^t$, and
\pause \item a polarization of $A$ corresponds to a $\lambda\in K^\times$ such that
      \begin{enumerate}[-]
       \item $\lambda I \subseteq \overline{I}^t$ (isogeny);
       \item $\lambda$ is totally imaginary ($\overline \lambda = -\lambda$);
       \item $\lambda$ is $\Phi$-positive, where $\Phi$ is a specific CM-type of $K$.
% 	     \[\left(
% 	     \begin{split}
% 	     \vphi(\lambda)/\mathrm{i} >0,\forall \vphi\in \Phi:= & \set{ \vphi:\Z[F,V]\otimes \Q \to \C : v_{p,\epsilon}(\vphi(F))>0 },\\
% 	     & \overline \Q_p \overset{\epsilon}{\simeq} \C
% 	     \end{split}
% 	     \right)\]
      \end{enumerate}     
\pause  \item if $A \leftrightarrow I$ and $S=(I:I)$ then
  \[\set{\parbox[p]{7em}{non-isomorphic princ.~pol.'s of $A$}} \longleftrightarrow \dfrac{\set{\text{totally positive }u\in S^\times }}{\set{v\overline{v}: v\in S^\times}}\]
  and $\Aut(A,\lambda) = \set{\text{torsion units of $S$}}$
\end{itemize}
\end{frame}

\begin{frame}{ Example : Elliptic curves }
 For elliptic curves the number of isomorphism classes can be expressed as a closed formula (Deuring, Waterhouse).
 
 Let $h(x)=x^2+\beta x +q$, with $q=p^r$ and $\beta$ an integer coprime with $p$ such that $\beta^2<4q$.
 
 Put $F:=x \mod (h(x))$ in $K:=\Q[x]/(h)$.
 
 Then $\Z[F]=\Z[F,q/F]$ and
 \[\ICM(\Z[F]) = \bigsqcup_{n|f} \Pic(\Z+n\cO_K) \]
 where $f:=\#(\cO_K:\Z[F])$, which implies that
 \[
 \#\set{\parbox[p]{10em}{ iso.~classes of ell.~curves with $q-1+\beta$ $\F_q$-points }} =
 \dfrac{\#\Pic(\cO_K)}{\#\cO_K^\times} \sum_{n|f} n\prod_{p|n}\left( 1-\dfrac{\Delta_K}{p}\dfrac{1}{p} \right) \]
\end{frame}

\begin{frame}{ Example : higher dimension }
\begin{itemize}
 \item Let $h(x)=x^8 - 5x^7 + 13x^6 - 25x^5 + 44x^4 - 75x^3 + 117x^2 - 135x + 81$;
 \item $\rightsquigarrow$ isogeny class of an simple ordinary abelian varieties over $\F_{3}$ of dimension $4$;
 \item Let $\alpha$ be a root of $h(x)$ and put $R:=\Z[\alpha,3/\alpha]\subset \Q(\alpha)$;
 \item $8$ over-orders of $R$: two of them are not Gorenstein;
 \item $\#\ICM(R) = 18 \rightsquigarrow 18$ isom.~classes of AV in the isogeny class;
 \item $5$ are not invertible in their multiplicator ring;
 \item $8$ classes admit principal polarizations;
 \item $10$ isomorphism classes of princ. polarized AV.
\end{itemize}
\end{frame}
\begin{frame}{Example}
Concretely:
{\scriptsize \begin{align*}
  \begin{split} 
  I_1 = & 2645633792595191 \Z \oplus (\alpha + 836920075614551) \Z \oplus (\alpha^2 + 1474295643839839)\Z \oplus\\
	& \oplus (\alpha^3 + 1372829830503387)\Z \oplus (\alpha^4 + 1072904687510)\Z \oplus\\
	& \oplus \frac{1}{3}(\alpha^5 + \alpha^4 + \alpha^3 + 2\alpha^2 + 2\alpha + 6704806986143610)\Z \oplus\\
	& \oplus \frac{1}{9}(\alpha^6 + \alpha^5 + \alpha^4 + 8\alpha^3 + 2\alpha^2 + 2991665243621169) \Z \oplus\\
	& \oplus \frac{1}{27}(\alpha^7 + \alpha^6 + \alpha^5 + 17\alpha^4 + 20\alpha^3 + 9\alpha^2 + 68015312518722201)\Z\\
  \end{split}
\intertext{principal polarizations:}
  \begin{split}
  x_{1,1} = \frac{1}{27}( & -121922\alpha^7 + 588604\alpha^6 - 1422437\alpha^5 +\\
			  & +1464239\alpha^4 + 1196576\alpha^3 - 7570722\alpha^2 + 15316479\alpha - 12821193)\\ 
%   \end{split}\\
%   \begin{split}
  x_{1,2} = \frac{1}{27}( & 3015467\alpha^7 - 17689816\alpha^6 + 35965592\alpha^5 -\\
			  & -64660346\alpha^4 + 121230619\alpha^3 - 191117052\alpha^2 + 315021546\alpha - 300025458)\\
  \end{split}\\
  & \End(I_1) =  R\\
  & \#\Aut(I_1,x_{1,1}) = \#\Aut(I_1,x_{1,2}) = 2
 \end{align*}}
\end{frame}


\begin{frame}{Example}
 
{\scriptsize \begin{align*}
  \begin{split} 
  I_7 = & 2\Z\oplus(\alpha + 1)\Z\oplus(\alpha^2 + 1)\Z\oplus(\alpha^3 + 1)\Z\oplus(\alpha^4 + 1)\Z\oplus(1/3(\alpha^5 + \alpha^4 + \alpha^3 + 2\alpha^2 + 2\alpha + 3)\Z \oplus \\ 		      & \oplus\frac{1}{36}(\alpha^6 + \alpha^5 + 10\alpha^4 + 26\alpha^3 + 2\alpha^2 + 27\alpha + 45)\Z\oplus\\
	& \oplus \frac{1}{216}(\alpha^7 + 4\alpha^6 + 49\alpha^5 + 200\alpha^4 + 116\alpha^3 + 105\alpha^2 + 198\alpha + 351)\Z\\
  \end{split}
\intertext{principal polarization:}\\[-7ex]
  \begin{split}
  x_{7,1} = \frac{1}{54}(20\alpha^7 - 43\alpha^6 + 155\alpha^5 - 308\alpha^4 + 580\alpha^3 - 1116\alpha^2 + 2205\alpha - 1809)
  \end{split}\\
  \begin{split}
  \End(I_7) & = \Z \oplus  \alpha\Z \oplus  \alpha^2\Z \oplus  \alpha^3\Z \oplus  \alpha^4\Z \oplus
  \frac{1}{3}(\alpha^5 + \alpha^4 + \alpha^3 + 2\alpha^2 + 2\alpha)\Z \oplus \\
	& \oplus \frac{1}{18}(\alpha^6 + \alpha^5 + 10\alpha^4 + 8\alpha^3 + 2\alpha^2 + 9\alpha + 9)\Z \oplus\\
	& \oplus \frac{1}{108}(\alpha^7 + 4\alpha^6 + 13\alpha^5 + 56\alpha^4 + 80\alpha^3 + 33\alpha^2 + 18\alpha + 27)\Z\\
  \end{split}\\
  & \#\Aut(I_7,x_{7,1}) = 2
\end{align*}}             
$I_1$ is invertible in $R$, but $I_7$ is not invertible in $\End(I_7)$.
\end{frame}

% \begin{frame}{ }
% \begin{center}
% Thank you for the attention 
% \end{center}
% \end{frame}

\end{document}