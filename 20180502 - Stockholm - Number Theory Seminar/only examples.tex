\documentclass{beamer}
% \documentclass[handout]{beamer}


% \usetheme{CambridgeUS}
\usetheme{Pittsburgh}
\usecolortheme{wolverine}
% \usetheme{Montpellier}
\usepackage{commands}
\usepackage{faktor}
\usepackage{xfrac} 

%AUTHOR DETAILS
%%%%%%%%%%%%%%%%%%%%%%%%%%%%%%%%%%%%%%%%%%%%%%%%
\title[]{Computing isomorphism classes of abelian varieties over finite fields}
\subtitle{Stockholm Number Theory Seminar}
\author[Marseglia Stefano]{Marseglia Stefano}
\institute[]{Stockholms University}
\date{02 May 2018}

\begin{document}

% \begin{frame}
% \titlepage
% \end{frame}
 

\begin{frame}{ back to AV's: Dual variety/Polarization }
\begin{itemize}
 \item Howe ('95) defined a notion of \textbf{dual} module and of \textbf{polarization} in the category of Deligne modules.
%  \item Fix an $\overline \Q_p \overset{\epsilon}{\simeq} \C$ and choose a \textbf{CM-type} as follows:
%  \[\Phi:=\set{ \vphi:\Z[F,V]\otimes \Q \to \C : v_{p,\epsilon}(\vphi(F))>0 }\]
\pause \item Concretely, if $A\leftrightarrow I$, then $A^\vee \leftrightarrow \overline{I}^t$, and
\pause \item a polarization $\mu$ of $A$ corresponds to a $\lambda\in K^\times$ such that
      \begin{enumerate}[-]
       \item $\lambda I \subseteq \overline{I}^t$ (isogeny);
       \item $\lambda$ is totally imaginary ($\overline \lambda = -\lambda$);
       \item $\lambda$ is $\Phi$-positive, where $\Phi$ is a specific CM-type of $K$.
% 	     \[\left(
% 	     \begin{split}
% 	     \vphi(\lambda)/\mathrm{i} >0,\forall \vphi\in \Phi:= & \set{ \vphi:\Z[F,V]\otimes \Q \to \C : v_{p,\epsilon}(\vphi(F))>0 },\\
% 	     & \overline \Q_p \overset{\epsilon}{\simeq} \C
% 	     \end{split}
% 	     \right)\]      
      \end{enumerate} 
      Also: $\deg \mu= [\overline{I}^t : I]$.
\pause  \item if $A \leftrightarrow I$ and $S=(I:I)$ then
  \[\set{\parbox[p]{7.5em}{non-isomorphic polarizations of $A$}} \longleftrightarrow \dfrac{\set{\text{totally positive }u\in S^\times }}{\set{v\overline{v}: v\in S^\times}}\]
  and $\Aut(A,\mu) = \set{\text{torsion units of $S$}}$
\end{itemize}
\end{frame}

\begin{frame}{ Example : Elliptic curves }
 For elliptic curves the number of isomorphism classes can be expressed as a closed formula (Deuring, Waterhouse).
 
 Let $h(x)=x^2+\beta x +q$, with $q=p^r$ and $\beta$ an integer coprime with $p$ such that $\beta^2<4q$.
 
 Put $F:=x \mod (h(x))$ in $K:=\Q[x]/(h)$.
 
 Then $\Z[F]=\Z[F,q/F]$ and
 \[\ICM(\Z[F]) = \bigsqcup_{n|f} \Pic(\Z+n\cO_K) \]
 where $f:=[\cO_K:\Z[F]]$, which implies that
 \[
 \#\set{\parbox[p]{10em}{ iso.~classes of ell.~curves with $q-1+\beta$ $\F_q$-points }} =
 \dfrac{\#\Pic(\cO_K)}{\#\cO_K^\times} \sum_{n|f} n\prod_{p|n}\left( 1-\dfrac{\Delta_K}{p}\dfrac{1}{p} \right) \]
\end{frame}

\begin{frame}{ Example}
\begin{itemize}
 \item Let $h(x)=x^8 - 5x^7 + 13x^6 - 25x^5 + 44x^4 - 75x^3 + 117x^2 - 135x + 81$;
 \item $\rightsquigarrow$ isogeny class of an simple ordinary abelian varieties over $\F_{3}$ of dimension $4$;
 \item Let $F$ be a root of $h(x)$ and put $R:=\Z[F,3/F]\subset \Q(F)$;
 \item $8$ over-orders of $R$: two of them are not Gorenstein;
 \item $\#\ICM(R) = 18 \rightsquigarrow 18$ isom.~classes of AV in the isogeny class;
 \item $5$ are not invertible in their multiplicator ring;
 \item $8$ classes admit principal polarizations;
 \item $10$ isomorphism classes of princ. polarized AV.
\end{itemize}
\end{frame}
\begin{frame}{Example}
Concretely:
{\scriptsize \begin{align*}
  \begin{split} 
  I_1 = & 2645633792595191 \Z \oplus (F + 836920075614551) \Z \oplus (F^2 + 1474295643839839)\Z \oplus\\
	& \oplus (F^3 + 1372829830503387)\Z \oplus (F^4 + 1072904687510)\Z \oplus\\
	& \oplus \frac{1}{3}(F^5 + F^4 + F^3 + 2F^2 + 2F + 6704806986143610)\Z \oplus\\
	& \oplus \frac{1}{9}(F^6 + F^5 + F^4 + 8F^3 + 2F^2 + 2991665243621169) \Z \oplus\\
	& \oplus \frac{1}{27}(F^7 + F^6 + F^5 + 17F^4 + 20F^3 + 9F^2 + 68015312518722201)\Z\\
  \end{split}
\intertext{principal polarizations:}
  \begin{split}
  x_{1,1} = \frac{1}{27}( & -121922F^7 + 588604F^6 - 1422437F^5 +\\
			  & +1464239F^4 + 1196576F^3 - 7570722F^2 + 15316479F - 12821193)\\ 
%   \end{split}\\
%   \begin{split}
  x_{1,2} = \frac{1}{27}( & 3015467F^7 - 17689816F^6 + 35965592F^5 -\\
			  & -64660346F^4 + 121230619F^3 - 191117052F^2 + 315021546F - 300025458)\\
  \end{split}\\
  & \End(I_1) =  R\\
  & \#\Aut(I_1,x_{1,1}) = \#\Aut(I_1,x_{1,2}) = 2
 \end{align*}}
\end{frame}


\begin{frame}{Example}
 
{\scriptsize \begin{align*}
  \begin{split} 
  I_7 = & 2\Z\oplus(F + 1)\Z\oplus(F^2 + 1)\Z\oplus(F^3 + 1)\Z\oplus(F^4 + 1)\Z\oplus\frac13(F^5 + F^4 + F^3 + 2F^2 + 2F + 3)\Z \oplus \\ 		      & \oplus\frac{1}{36}(F^6 + F^5 + 10F^4 + 26F^3 + 2F^2 + 27F + 45)\Z\oplus\\
	& \oplus \frac{1}{216}(F^7 + 4F^6 + 49F^5 + 200F^4 + 116F^3 + 105F^2 + 198F + 351)\Z\\
  \end{split}
\intertext{principal polarization:}\\[-7ex]
  \begin{split}
  x_{7,1} = \frac{1}{54}(20F^7 - 43F^6 + 155F^5 - 308F^4 + 580F^3 - 1116F^2 + 2205F - 1809)
  \end{split}\\
  \begin{split}
  \End(I_7) & = \Z \oplus  F\Z \oplus  F^2\Z \oplus  F^3\Z \oplus  F^4\Z \oplus
  \frac{1}{3}(F^5 + F^4 + F^3 + 2F^2 + 2F)\Z \oplus \\
	& \oplus \frac{1}{18}(F^6 + F^5 + 10F^4 + 8F^3 + 2F^2 + 9F + 9)\Z \oplus\\
	& \oplus \frac{1}{108}(F^7 + 4F^6 + 13F^5 + 56F^4 + 80F^3 + 33F^2 + 18F + 27)\Z\\
  \end{split}\\
  & \#\Aut(I_7,x_{7,1}) = 2
\end{align*}}             
$I_1$ is invertible in $R$, but $I_7$ is not invertible in $\End(I_7)$.
\end{frame}

% \begin{frame}{ Final remarks }
% \begin{itemize}
% %    \item the same correspondence between iso.~classes of AV's and $\ICM$ holds for isogeny classes $\cC_h$ over $\F_p$ with $h(\sqrt{p})\neq 0$ square-free\\
% % \pause much larger subcategory ... but no polarizations in this case.
% % 	  (Centeleghe and Stix 2015)
%          \item Using Centeleghe-Stix '15 we can compute the isomorphism classes in $\cC_h$ over $\F_p$  where $h$ is square-free and $h(\sqrt{p})\neq 0$\\
% \pause much larger subcategory ... but no polarizations in this case.         
% \pause   \item we can also deal with the case $\cC_{h^d}$ (with $h$ square-free) when $\Z[F,q/F]$ is Bass.
% \pause   \item base field extensions (ordinary case).
% \pause   \item period matrices (ordinary case) of the canonical lift.
% % \pause   \item conjugacy classes of integral matrices.
% \end{itemize}
% \end{frame}

\begin{frame}{ }
\begin{center}
{\Large Thank you!}
\end{center}
\end{frame}

\end{document}
