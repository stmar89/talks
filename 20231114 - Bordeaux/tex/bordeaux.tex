\documentclass[usenames,dvipsnames]{beamer}
%\documentclass[usenames,dvipsnames,handout]{beamer}

\usetheme{AnnArbor}
% \usecolortheme{default}
% \usecolortheme{crane}
\usecolortheme{beaver}
\usecolortheme{dolphin}
% \usecolortheme{orchid}
% \usecolortheme{rose}


\usepackage{fourier}
\usepackage{faktor}
\usepackage{amssymb}
\usepackage{amsmath}
\usepackage{amsthm}
%\usepackage{stmaryrd}
\usepackage{hyperref}
\usepackage[all]{xy}
\usepackage{tikz}
%    \usetikzlibrary{mindmap,shadows,shapes.geometric,shapes.misc,positioning}
\usepackage{tikz-cd}
\tikzset{
    invisible/.style={opacity=0},
    visible on/.style={alt={#1{}{invisible}}},
    alt/.code args={<#1>#2#3}{%
      \alt<#1>{\pgfkeysalso{#2}}{\pgfkeysalso{#3}}%
  }
}
%\usetikzlibrary{matrix}
%\usetikzlibrary{calc,intersections}
%\newcommand{\downmapsto}{\rotatebox[origin=c]{-90}{$\large\mapsto$}\mkern2mu} %MnSymbol doesn't work well with beamer
\usepackage{multirow}
\usepackage{pbox}
\newcommand{\pboxc}[2]{\pbox{#1}{\relax\ifvmode\centering\fi#2}}
\usepackage{cellspace}
\setlength{\cellspacetoplimit}{2pt}
\setlength{\cellspacebottomlimit}{2pt}

\def\Q{\mathbb{Q}}
\def\Z{\mathbb{Z}}
\def\C{\mathbb{C}}
\def\R{\mathbb{R}}
\def\F{\mathbb{F}}

\DeclareMathOperator{\AV}{AV}
\DeclareMathOperator{\Mat}{Mat}
\DeclareMathOperator{\Pol}{Pol}
\DeclareMathOperator{\Char}{char}
\DeclareMathOperator{\rk}{Rank}
\DeclareMathOperator{\Frob}{Frob}
\DeclareMathOperator{\ICM}{ICM}
\DeclareMathOperator{\Pic}{Pic}
\DeclareMathOperator{\Aut}{Aut}
\DeclareMathOperator{\Hom}{Hom}
\DeclareMathOperator{\End}{End}
\DeclareMathOperator{\Gal}{Gal}
\DeclareMathOperator{\mSpec}{mSpec}
\DeclareMathOperator{\GL}{GL}
\DeclareMathOperator{\Tr}{Tr}
\DeclareMathOperator{\Jac}{Jac}
%\renewcommand{\char}{char} %CRASHES WITH beamer


\newcommand{\cG}{\mathcal{G}}
%\newcommand{\cB}{{\mathcal B}}
\newcommand{\cA}{{\mathcal A}}
\newcommand{\cC}{{\mathcal C}}
\newcommand{\cO}{{\mathcal O}}
\newcommand{\cH}{{\mathcal H}}
%\newcommand{\cM}{{\mathcal M}}
\newcommand{\cT}{{\mathcal T}}
\newcommand{\cW}{{\mathcal W}}


\newcommand{\vphi}{\varphi}

\newcommand{\p}{{\mathfrak p}}
\newcommand{\frf}{{\mathfrak f}}

\newcommand{\downmapsto}{\rotatebox[origin=c]{-90}{$\large\mapsto$}\mkern2mu} %MnSymbol doesn't work well with beamer
\newcommand{\set}[1]{\left\lbrace#1\right\rbrace }
\newcommand{\Span}[1]{\left<#1\right>}

%\newcommand{\AVord}[1]{\AV^{\text{ord}}({#1})}
%\newcommand{\Modord}[1]{\cM^{\text{ord}}({#1})}

%\newcommand{\AVcs}[1]{\AV^{\text{cs}}({#1})}
%\newcommand{\Modcs}[1]{\cM^{\text{cs}}({#1})}

\newcommand{\Palpha}[2]{\mathcal{P}^{\alpha}_{{#1}}({#2})}
\newcommand{\Pone}[2]{\mathcal{P}^{1}_{{#1}}({#2})}

\newcommand{\red}[1]{\textcolor{red}{#1}}
\newcommand{\blue}[1]{\textcolor{blue}{#1}}
\newcommand{\green}[1]{\textcolor{ForestGreen}{#1}}

\newtheorem{df}{Definition}[section]
\newtheorem{remark}[df]{Remark}
\newtheorem{prop}[df]{Proposition}
\newtheorem{cor}[df]{Corollary}



%AUTHOR DETAILS
%%%%%%%%%%%%%%%%%%%%%%%%%%%%%%%%%%%%%%%%%%%%%%%%
\title[]{Computing isomorphism classes and polarisations of abelian varieties over finite fields.}
\subtitle{}
\author[Stefano Marseglia]{Stefano Marseglia}
\institute[]{Utrecht University}
\date[1 June 2021]{\small Séminaire de Théorie Algorithmique des Nombres\\Institut de Mathématiques de Bordeaux\\November 7, 2023
% \\\vspace*{2em} \pause joint work with\\ {\bf Jonas Bergstr\"om} and {\bf Valentijn Karemaker}.
}

\begin{document}

\begin{frame}
	\titlepage
\end{frame}

\begin{frame}{ Abelian varieties: what are they ? }
\vspace{-0.5cm}
\onslide<1->{Abelian varieties are connected projective group varieties.}
\begin{center}
	\begin{columns}
		\begin{column}{0.4\textwidth}
			\onslide<2->{Abelian varieties of dim.~$1$\\
							are called {\bf elliptic curves}.\\
							Eg: over $\R$, $\red{y^2 = x^3 -x +1}$
							\vspace{1em}}
				% \vpsace{1em}
			\onslide<3->{We can add points:\\
							$P,Q$ {\Large \blue{$\leadsto$}} $P\oplus Q$
							\vspace{1em}\\
							}
				\onslide<4->{Equations are impractical in $\dim \geq 2$.\\
							We need a better way to represent them...}             
		\end{column}
		\begin{column}{0.5\textwidth}
			\tikz{
				\onslide<2->{
				\draw [help lines] (-2,-2.24) grid (1.8,2.24);
				\draw [->] (-2-0.2,0) -- (1.8+0.2,0) node[right] {$x$}; 
				\draw [->] (0,-2.24-0.2) -- (0,2.24+0.2) node[left] {$y$}; 
				\draw [red, thick, domain=-1.32471795724474602596090885448:1.8, samples=100]
				plot (\x, {sqrt(\x*\x*\x -\x +1)});
				\draw [red, thick, domain=-1.32471795724474602596090885448:1.8, samples=100]
				plot (\x, -{sqrt(\x*\x*\x -\x +1)});
				}
				\onslide<2->{
				\draw [blue, thick, domain=-2:1.8, samples=100] plot (\x, {0.7*\x +0.5 });
				\draw [blue, thick, dashed] (1.3407,2.24) -- (1.3407,-2.24);
				\draw (-1.2858-0.2,-0.7*1.2858+0.5+0.1) node {$P$};
				\draw (0.43506,0.7*0.43506+0.5 +0.3) node {$Q$};
				\draw (1.3407+0.5,-0.7*1.3407-0.5) node {$P\oplus Q$};
				}
			}
		\end{column}
	\end{columns}
\end{center}
\end{frame}

\begin{frame}{ Abelian varieties over $\C$ vs $\F_q$ }    
	\begin{itemize}
		\item Let $A/\C$ be an abelian variety of dimension $g$. 
\pause
	\item Then $A(\C)$ is a {\bf torus}: $T:=\C^g/\Lambda$, where $\Lambda\simeq_\Z\Z^{2g}$.
\pause 
	\item Also, $T$ admits a non-degenerate {\bf Riemann form} $\longleftrightarrow$ polarization.
\pause
	\item The functor $ A \mapsto A(\C)$ induces an \blue{equivalence} of categories:
		\[
		\set{ \text{abelian varieties $/\C$} } \longleftrightarrow 
		\set{\parbox[c]{12.5em}{\center $\C^g/\Lambda$ with \green{$\Lambda\simeq \Z^{2g}$} admitting\\ a Riemann form}}.
		\]
\pause
	\item In \red{char.~$p>0$} \green{such} an equivalence \red{cannot exist} : there are (supersingular) elliptic curves with quaternionic endomorphism algebras.
\pause 
	\item Nevertheless, over finite fields, we obtain \green{analogous} results if we restrict ourselves to certain {\bf subcategories} of AVs.
	\end{itemize}
\end{frame}

\begin{frame}{Isogeny classification over $\F_q$}
	\begin{itemize}
	\item $A/\F_{q}$ comes with a {\bf Frobenius} endomorphism, 
\pause
	that induces an action
		\[ \Frob_A : T_\ell A \rightarrow T_\ell A \text{ for any }\ell\neq p, \]
		where $T_\ell(A) = \varprojlim A[\ell^n] \simeq \Z_\ell^{2g}$.
\pause 
	\item $\blue{h_A(x)}:=\Char(\Frob_A)$ is a \blue{$q$-Weil} polynomial and {\bf isogeny invariant}.
\pause
	\item By {\bf Honda-Tate} theory, the association
		\[ A \longmapsto h_A(x) \]
		is injective up to isogeny and allows us to \blue{enumerate} all AVs up to isogeny.
\pause
	\item Also, $h_A(x)$ is squarefree $\iff$ $\End(A)$ is commutative.
	\end{itemize}
\end{frame}

\begin{frame}{ Deligne's equivalence }
	Recall: $A/\F_q$ is {\bf ordinary} if half of the $p$-adic roots of $h_A$ are units.
\pause
	\begin{theorem}[Deligne '69]
	Let $q=p^r$, with $p$ a prime.
	There is an \blue{equivalence} of categories:
	\[ \begin{array}{cc}
	\set{\text{ {\bf Ordinary} abelian varieties over } \F_q } 	& A \\
\pause
	\updownarrow											& \downmapsto \\
	\set{\parbox[p]{19em}{pairs $(T,F)$, where $T\simeq_\Z \Z^{2g}$ and $T\overset{F}{\to} T$ s.t.\\
\pause
	- $F\otimes \Q$ is semisimple\\
	- the roots of $\Char_{F\otimes\Q}(x)$ have abs. value $\sqrt{q}$\\
	- \textbf{half of them are $p$-adic units}\\
	- $\exists V:T\to T$ such that $FV=VF=q$
	}}	& (T(A),F(A))
	\end{array} \]
	\end{theorem}
\pause
	\begin{itemize}
		\item Ordinary $A/\F_q$ can be canonically lifted: $\leadsto \cA_{\mathrm{can}}/\mathrm{Witt}(\F_q)$...
\pause
		\item ... characterized by: $\End_{\F_q}(A) = \End_{\mathrm{Witt}(\F_q)}(\cA_{\mathrm{can}})$.
\pause
		\item Put $T(A):=H_1(\cA_{\mathrm{can}}\otimes \C,\Z) $
\pause
		and $F(A):=$ the induced Frobenius.
	\end{itemize}
\end{frame}

\begin{frame}{Squarefree case}
	\begin{itemize}
	\item Fix an \textbf{ordinary squarefree} $q$-Weil polynomial \blue{$h$} :
\pause  
	\item  $\rightsquigarrow \text{an isogeny class } \blue{\cC_h}$/$\F_q$.
\pause 
	\item Put $K := \Q[x]/(h)=\Q[F]$, an \'etale algebra = product of number fields.
\pause
	\item Put $V=q/F$. Deligne's equivalence induces:
\pause
			\begin{theorem}
			\[\begin{array}{cc}
			\faktor{\set{\text{abelian varieties over $\F_q$ in $\cC_h$}}}{\simeq} & \\
			\updownarrow & \\
			\faktor{\set{ \text{fractional ideals of $\Z[F,V] \subset K$ } }}{\simeq} &
\pause =:  \blue{\ICM(\Z[F,V])}\\ 
			& \blue{\text{ ideal class monoid} }
				\end{array}\]
			\end{theorem}
\pause
	\item \red{Problem:} $\Z[F,V]$ might not be maximal $\rightsquigarrow $ \red{non-invertible} ideals.
	\end{itemize}
\end{frame}

\begin{frame}{ICM : Ideal Class Monoid}
	Let $R$ be an {\bf order} in an \'etale  $\Q$-algebra $K$.
% according to Bourbaki etale (over a field K) implies commutative and finite (since it is defined as being isomorphic to L^n for some extension L of K).
	\begin{itemize}
\pause
	\item Recall: for {\bf fractional $R$-ideals} $I$ and $J$
		\[ I\simeq_R J \Longleftrightarrow \exists x \in K^\times \text{ s.t.~} xI=J \]
\pause
	\item We have
		\begin{align*}
	\ICM(R) & \supseteq \Pic(R)=\faktor{\set{\text{invertible fractional $R$-ideals}}}{\simeq_R} \\
	\blue{\text{with equality }} & \blue{\rotatebox[origin=c]{90}{$\large\rightsquigarrow$}\mkern2mu \text{ iff } R=\cO_K }
	\end{align*}
\pause 
	\item ...and actually
	\[ \ICM(R) \supseteq \bigsqcup_{\scriptsize\parbox{5 em}{$R\subseteq S \subseteq \cO_K$\\over-orders}}\Pic(S) \qquad   
		\textcolor{blue}{\text{with equality iff $R$ is Bass}} \]
\pause
	\item Hofmann-Sircana : computation of over-orders.
\end{itemize}
\end{frame}

\begin{frame}{ simplify the problem  }
	First, locally: Dade-Taussky-Zassenhaus.
	\begin{itemize}
\pause 
	\item  \textbf{weak equivalence}:
	\[I_{\p}\simeq_{R_{\p}} J_{\p} \text{ for every } {\p} \in \mSpec(R)\]
\pause
	\vspace{-6mm}\[\Updownarrow\]
	\[1\in (I:J)(J:I)\quad \textcolor{blue}{\text{easy to check!}}\]
\pause
	\item Let $\cW(R)$ be the set of weak eq.~classes...\\
\pause
	...whose representatives can be found in
	\[\set{\text{sub-$R$-modules of $\faktor{\cO_K}{\frf_R}$ }} \quad \textcolor{blue}{\parbox{10 em}{finite! and most of the time not-too-big ...}}\]
	where $\frf_R=(R:\cO_K)$ is the conductor of $R$.
	\end{itemize}
\end{frame}

\begin{frame}{ Compute $\ICM(R)$ }
\pause 
	Partition w.r.t. the multiplicator ring:
	\begin{columns}
	\begin{column}{0.5\textwidth}
		\[ \cW(R) = \bigsqcup_{R\subseteq S \subseteq \cO_K} \cW_S(R)\]
		\[ \ICM(R) = \bigsqcup_{R\subseteq S \subseteq \cO_K} \ICM_S(R)\]
	\end{column}
\pause
	\begin{column}{0.5\textwidth}  %%<--- here
	\begin{center}
	\textcolor{blue}{\parbox{10em}{the ``pedix'' $-_S$ means ``only classes with multiplicator ring S''}} 
	\end{center}
	\end{column}
	\end{columns}
\pause
	\begin{theorem}[M.]
	For every over-order $S$ of $R$, $\Pic(S)$ acts \red{freely} on $\ICM_S(R)$ and
	\[ \cW_S(R) = \ICM_S(R) / \Pic(S) \]
\pause
	Repeat for every $R\subseteq S \subseteq \cO_K$:
	\[ \rightsquigarrow \ICM(R).\]
	\end{theorem}
\end{frame}

\begin{frame}{ To sum up: }
	\begin{itemize}
	\item To sum up:
\pause
	\item Given a {\bf ordinary squarefree} $q$-Weil polynomial $h$ ...
\pause
	\item ... $\leadsto$ algorithm to \blue{compute the isomorphism classes} of AVs in~$\cC_h$.
\pause    
	\item What about polarizations?
	\end{itemize}
\end{frame}

\begin{frame}{Dual varieties and Polarizations }
	Howe : \textbf{dual} varieties and \textbf{polarizations} on Deligne modules.
\pause
	\begin{theorem}[M.]\
	Let $A\in \cC_h$ with $h$ ordinary and squarefree. If $A\leftrightarrow I$, set $S=(I:I)$, then:
	\begin{itemize}
\pause
	\item $A^\vee \leftrightarrow \overline{I}^t:=\set{ \overline{x} \in K : \Tr(xI)\subseteq \Z}$.
\pause
	\item a polarization $\mu$ of $A$ corresponds to a $\lambda\in K^\times$ such that\\
	- $\lambda I \subseteq \overline{I}^t$ (isogeny of $\deg \mu= [\overline{I}^t : \lambda I]$);\\
	- $\lambda$ is \blue{totally imaginary} ($\overline \lambda = -\lambda$);\\
	- $\lambda$ is $\Phi$-positive ($\Im\vphi(\lambda)>0$ for all $\vphi\in \Phi$),\\
		where $\Phi$ is a CM-type of $K$ satisf.~the \blue{Shimura-Taniyama} formula.\\ 
\pause
	\item if $(A,\mu) \leftrightarrow (I,\lambda)$ is a princ.~polarized ab.~var.~and $S=(I:I)$ then
	\vspace{-0.8em}
	\[\set{\parbox[p]{9em}{non-isomorphic princ. polarizations of $A$}} \longleftrightarrow \dfrac{\set{\text{totally positive }u\in S^\times }}{\set{v\overline{v}: v\in S^\times}},\
\pause
	\pbox{2cm}{\centering \green{similar statement for $\deg\mu>1$}}   
	\]
	\vspace{-1.7em}
\pause
	\item  and $\Aut(A,\mu) = \set{\text{torsion units of $S$}}$.
	\end{itemize}
	\end{theorem}
\end{frame}


\begin{frame}{Principal Polarizations}
	We have an {\bf algorithm} to enumerate principal polarizations up to isomorphism:
\pause
	\begin{enumerate}
	\item Compute $i_0$ such that $i_0 I = \overline{I}^t$.
\pause
	\item Loop over the representatives $u$ of the finite quotient
	\[ \frac{S^\times}{\set{v\overline{v}: v\in S^\times}}. \]
\pause    
	\item If $\lambda:=i_0 u$ is totally imaginary and $\Phi$-positive ...
\pause    
	\item ... then we have one principal polarization.
\pause    
	\item By the previous Theorem, we have all princ.~polarizations up to isom.
	\end{enumerate}
%\pause 
%    Can modify to compute polarizations of any degree.
\end{frame}

\begin{frame}{Example}
	\begin{itemize}
	\item Let $h(x)=x^8 - 5x^7 + 13x^6 - 25x^5 + 44x^4 - 75x^3 + 117x^2 - 135x + 81$.
\pause
	\item $\rightsquigarrow$ isogeny class of an simple ordinary abelian varieties over $\F_{3}$ of dimension $4$.
\pause
	\item Let $F$ be a root of $h(x)$ and put $R:=\Z[F,3/F]\subset \Q(F)$.
\pause
	\item $8$ over-orders of $R$: two of them are not Gorenstein.
\pause
	\item $\#\ICM(R) = 18 \rightsquigarrow 18$ isom.~classes of AV in the isogeny class.
\pause
	\item $5$ are not invertible in their multiplicator ring.
\pause
	\item $8$ classes admit principal polarizations.
\pause
	\item $10$ isomorphism classes of princ. polarized AV.
	\end{itemize}
\end{frame}

\begin{frame}{Example}{}
	Concretely:
	{\scriptsize \begin{align*}
	\begin{split} 
	I_1 = & 2645633792595191 \Z \oplus (F + 836920075614551) \Z \oplus (F^2 + 1474295643839839)\Z \oplus\\
	& \oplus (F^3 + 1372829830503387)\Z \oplus (F^4 + 1072904687510)\Z \oplus\\
	& \oplus \frac{1}{3}(F^5 + F^4 + F^3 + 2F^2 + 2F + 6704806986143610)\Z \oplus\\
	& \oplus \frac{1}{9}(F^6 + F^5 + F^4 + 8F^3 + 2F^2 + 2991665243621169) \Z \oplus\\
	& \oplus \frac{1}{27}(F^7 + F^6 + F^5 + 17F^4 + 20F^3 + 9F^2 + 68015312518722201)\Z\\
	\end{split}
	\intertext{principal polarizations:}
	\begin{split}
	x_{1,1} = \frac{1}{27}( & -121922F^7 + 588604F^6 - 1422437F^5 +\\
				& +1464239F^4 + 1196576F^3 - 7570722F^2 + 15316479F - 12821193)\\ 
	%   \end{split}\\
	%   \begin{split}
	x_{1,2} = \frac{1}{27}( & 3015467F^7 - 17689816F^6 + 35965592F^5 -\\
				& -64660346F^4 + 121230619F^3 - 191117052F^2 + 315021546F - 300025458)\\
		\end{split}\\
	& \End(I_1) =  R\\
	& \#\Aut(I_1,x_{1,1}) = \#\Aut(I_1,x_{1,2}) = 2
	\end{align*}}
\end{frame}


\begin{frame}{Example}{}
	{\scriptsize \begin{align*}
	\begin{split} 
	I_7 = & 2\Z\oplus(F + 1)\Z\oplus(F^2 + 1)\Z\oplus(F^3 + 1)\Z\oplus(F^4 + 1)\Z\oplus\frac13(F^5 + F^4 + F^3 + 2F^2 + 2F + 3)\Z \oplus \\ 		      & \oplus\frac{1}{36}(F^6 + F^5 + 10F^4 + 26F^3 + 2F^2 + 27F + 45)\Z\oplus\\
	& \oplus \frac{1}{216}(F^7 + 4F^6 + 49F^5 + 200F^4 + 116F^3 + 105F^2 + 198F + 351)\Z\\
	\end{split}
	\intertext{principal polarization:}\\[-7ex]
	\begin{split}
	x_{7,1} = \frac{1}{54}(20F^7 - 43F^6 + 155F^5 - 308F^4 + 580F^3 - 1116F^2 + 2205F - 1809)
	\end{split}\\
	\begin{split}
	\End(I_7) & = \Z \oplus  F\Z \oplus  F^2\Z \oplus  F^3\Z \oplus  F^4\Z \oplus
	\frac{1}{3}(F^5 + F^4 + F^3 + 2F^2 + 2F)\Z \oplus \\
	& \oplus \frac{1}{18}(F^6 + F^5 + 10F^4 + 8F^3 + 2F^2 + 9F + 9)\Z \oplus\\
	& \oplus \frac{1}{108}(F^7 + 4F^6 + 13F^5 + 56F^4 + 80F^3 + 33F^2 + 18F + 27)\Z\\
	\end{split}\\
	& \#\Aut(I_7,x_{7,1}) = 2
	\end{align*}}             
	$I_1$ is invertible in $R$, but $I_7$ is not invertible in $\End(I_7)$.
\end{frame}

\begin{frame}{ Outside of the ordinary... }
\pause
	\begin{theorem}[Centeleghe-Stix~'15]
	There is an equivalence of categories:
	\[ \begin{array}{cc}
	\set{\text{abelian varieties $A$ over \red{$\F_p$} with $\pmb{h_A(\sqrt{p})\neq 0}$}} 	& A \\
	\updownarrow											& \downmapsto \\
	\set{\parbox[p]{19em}{pairs $(T,F)$, where $T\simeq_\Z \Z^{2g}$ and $T\overset{F}{\to} T$ s.t.\\
	- $F\otimes \Q$ is semisimple\\
	- the roots of $\Char_{F\otimes\Q}(x)$ have abs. value \red{$\sqrt{p}$}\\
	- $\pmb{\Char_{F}(\sqrt{p})\neq 0}$\\
	- $\exists V:T\to T$ such that $FV=VF=\red{p}$
	}}	& (T(A),F(A))
	\end{array} \]
	\end{theorem}
\pause
	\begin{itemize}
		\item Now, $T(A):=\Hom(A,A_w)$, where $A_w$ has minimal $\End$ among the varieties with Weil support $w=w(A)$.
		\item $F(A)$ is the induced Frobenius.
	\end{itemize}
\end{frame}

\begin{frame}{ Outside of the ordinary...isomorphism classes } 
	\begin{itemize}
		\item Everything I told so far about \blue{isomorphism classes} works in the \blue{same way} using the Centeleghe-Stix functor.
	\pause
	%    \item (There are there are other functors:\\
	%\pause
	%          Oswal-Shankar for almost-ordinary abelian varieties, together with generalzations by Bergstr\"om-Karemaker-M.;\\
	%\pause          
	%          Serre and the work of Kani, Jordan-Keeton-Poonen-Rains-\\-Shepherd-Barron-Tate for $E^r$; \\
	%          \red{THE NEW CS functor!}\\
	%\pause
	%    ... but we will not use them in this talk.)
	\pause
		\item For polarizations, the results by Howe do \red{not apply immediately} to the Centeleghe-Strix case:
	\pause
		\item in general we \red{cannot} lift canonically \red{each} abelian variety.
	\end{itemize}
\end{frame}

\begin{frame}{ Outside of the ordinary...polarizations } 
	\begin{itemize}
	\item New strategy: jt.~Jonas Bergstr\"om and Valentijn Karemaker.
\pause
	\item Consider $\cC_h$ with $h$ squarefree $/\F_q$ $\leadsto K=\Q[F]$. 
\pause 
	\item Chai-Conrad-Oort:
	A ($p$-adic) CM-type $(K,\Phi)$ satisfies the {\bf Residual Reflex Condition}~if:
	\begin{enumerate}[1.]
\pause 
	\item the \blue{Shimura-Taniyama formula} holds for $\Phi$.
%	    : for every  place $\nu$ of $L$ above~$p$, we have
%		\begin{equation*}
%		\dfrac{ \mathrm{ord}_\nu(F)}{ \mathrm{ord}_\nu(q)}=\dfrac{\#\set{ \vphi \in \Phi \text{ s.t.~} \vphi \text{ induces } \nu }}{[L_\nu:\Q_p]}.
%		\end{equation*}
\pause
	\item the residuel field $k_E$ of the reflex field $E$ of $(K,\Phi)$ satisfies: \blue{$k_E \subseteq \F_q$}.
	\end{enumerate}
\pause
	\begin{theorem}[Chai-Conrad-Oort]
	If $(K,\Phi)$ satisfies the RRC then in $\cC_h$ there exists an abelian variety $A$ admitting a canonical lifting $\cA$. 
	\end{theorem}
\pause
	\item If we understand the polarizations of $A$ we can 'spread' them to the whole isogeny class.
	\end{itemize}
\end{frame}

\begin{frame}{ Outside of the ordinary...polarizations }
\onslide<+->
	Now $\cC_h$ over $\F_p$: let $\cG$ be the Centeleghe-Stix functor.\\
	Assume that there exists $A$ admitting a canonical lifting $\cA$.\\
\onslide<4->{Let $f:A\to B$ be an isogeny.}  
	\[
	\begin{tikzcd}[ampersand replacement=\&] %needed for Beamer
	\& |[visible on=<2->]| \Hom(\cA,\cA^\vee) \arrow[visible on=<2->,d, "\mathrm{red}"]    \arrow[visible on=<2->,dr, "\text{complex unif.}"] \& \\
	|[visible on=<4->]|\Hom(B,B^\vee) \arrow[visible on=<4->,r,"f^*:=f^\vee\circ - \circ f"] \arrow[visible on=<5->,d, "\cG"] \& 
	|[visible on=<2->]| \Hom(A,A^\vee)\arrow[visible on=<3->,d,"\cG"] \&
	|[visible on=<2->]| (\overline I^t : I) \arrow[visible on=<3->,d,"\red{\alpha=\overline \alpha}","\simeq"'] \\
	|[visible on=<5->]| (\cG(B) : \cG(B^\vee))\arrow[visible on=<5->,r,"\cG(f^*)"] \&
	|[visible on=<3->]| (\cG(A) : \cG(A^\vee)) \& |[visible on=<3->]| (\overline I^t : I)
	\arrow[visible on=<3->,l,equal]
	\end{tikzcd}
	\]
\onslide<6->
	Note that $\cG(f^*)$ is multiplication by the totally positive element $\overline{\cG(f)}\cG(f)$:\\
\onslide<7->
	it sends totally imaginary elements to totally imaginary elements and $\Phi$-positive elements to $\Phi$-positive elements. 
\onslide<8->
	The only 'issue' is the \red{$\alpha$}.\\
\onslide<9->   
	We study when we can \blue{'pretend' $\alpha=1$}.
\end{frame}


\begin{frame}{ Effective Results : when can we ignore $\alpha$? }
    Assume $A$ admits a canonical lifting. Put $S:=\End(A)$\\
    Let $B$ be isogenous to $A$.
	Put $T=\End(B)$.
	The previous diagram tells us that the princ.~polarisations of $B$ (up-to-iso) are in bijections with
	\[ \Palpha{\Phi}{B}:=\{ i_0 \cdot u  : u \in T^\times/\set{v\overline v : v \in T^\times} \text{ s.t.~} \red{\alpha^{-1}} i_0 u \text{ is tot.~img.~and } \Phi\text{-pos.} \} \]
\pause      
	\vspace*{-2em}
	\begin{theorem}[ 1 ]
		Denote by $S^*_\R$ (resp.~$T^*_\R$) the group of totally real units of $S$ (resp.~$T$).\\
\pause 
		If $S^*_\R\subseteq T^*_\R$, then the set $\Palpha{\Phi}{B}$
\pause
	    is in bijection with the set
        \[ \Pone{\Phi}{B}=\{ i_0 \cdot u: u \in T^\times/\set{v\overline v : v \in T^\times} \text{ s.t.~} i_0 u \text{ is tot.~img.~and } \Phi\text{-pos.} \}. \]
		(which does not depend on $\alpha$!)
	\end{theorem}
\pause 
	\begin{cor}
    If $S=\Z[F,V]$ (eg. $\AV_h(p)$ is ordinary or almost-ordinary) then we can ignore $\alpha$.
\pause 
    \green{We recover Deligne+Howe and Oswal-Shankar}
	\end{cor}
\end{frame}

\begin{frame}{ }
	We run computations over all squarefree isogeny classes over small prime fields of dim $2,3$ and $4$. 
	\pause 
	\begin{table}[ht]
	    \centering
	    \small
	    \begin{tabular}{|c|c|c|c|c|c|c|}\hline
	\multicolumn{3}{|c|}{squarefree dimension $3$}                  & $p=2$ & $p=3$ & $p=5$ & $p=7$ \\\hline
	\multicolumn{3}{|c|}{total}                                     & $185$ & $621$  & $2863$ & $7847$ \\\hline                    
	\multicolumn{3}{|c|}{ordinary}                                  & $82$ & $390$  & $2280$  & $6700$ \\\hline
	\multicolumn{3}{|c|}{almost ordinary}                           & $58$ & $170$  & $474$  & $996$  \\\hline
	\multirow{3}{*}{$p$-rank $1$} & \multicolumn{2}{|c|}{no RRC}    & $0$ & $0$   & $0$   & $0$   \\\cline{2-7}
	                              & \multirow{2}{*}{yes RRC} & Thm 1 yes & $20$ & $26$   & $76$   & $118$   \\\cline{3-7}
	                              &                          & Thm 1 no  & $4$ & $16$   & $12$   & $8$   \\\hline
	\multirow{3}{*}{$p$-rank $0$}   & \multicolumn{2}{|c|}{no RRC}    & $0$ & $3$   & $2$   & $1$   \\\cline{2-7}
	                              & \multirow{2}{*}{yes RRC} & Thm 1 yes & $20$ & $15$   & $17$   & $23$   \\\cline{3-7}
	                              &                          & Thm 1 no  & $1$ & $1$   & $2$   & $1$   \\\hline                              
	    \end{tabular}
	%    \caption{Squarefree isogeny classes of dimension $3$. The notation is the same as in Table \ref{tab:dim2}.}
	%    \label{tab:dim3}
	\end{table}
    \pause Among the $45$ isogeny classes which we cannot 'handle' with Thm 1, we can compute the number of PPAV for $32$ of them using other techniques. For the remaining $13$ (all over $\F_2$ and $\F_3$) we only get partial info.
\end{frame}

\begin{frame}{ }{}
\begin{table}[ht]
    \centering
\footnotesize	
    \begin{tabular}{|c|c|c|c|c|}\hline
\multicolumn{3}{|c|}{squarefree dimension $4$}                  & $p=2$ & $p=3$ \\\hline
\multicolumn{3}{|c|}{total}                                     & $1431$ & $10453$  \\\hline                    
\multicolumn{3}{|c|}{ordinary}                                  & $656$ & $6742$  \\\hline
\multicolumn{3}{|c|}{almost ordinary}                           & $392$ & $2506$  \\\hline
\multirow{3}{*}{$p$-rank $2$} & \multicolumn{2}{|c|}{no RRC}    & $0$ & $0$ \\\cline{2-5}
                              & \multirow{2}{*}{yes RRC} & Thm 1 yes & $149$ & $500$   \\\cline{3-5}
                              &                          & Thm 1 no  & $49$ & $312$   \\\hline
\multirow{3}{*}{$p$-rank $1$} & \multicolumn{2}{|c|}{no RRC}    & $6$ & $36$ \\\cline{2-5}
                              & \multirow{2}{*}{yes RRC} & Thm 1 yes & $80$ & $184$   \\\cline{3-5}
                              &                          & Thm 1 no  & $14$ & $40$   \\\hline
\multirow{3}{*}{$p$-rank $0$} & \multicolumn{2}{|c|}{no RRC}    & $3$ & $6$   \\\cline{2-5}
                              & \multirow{2}{*}{yes RRC} & Thm 1 yes & $73$ & $88$ \\\cline{3-5}
                              &                          & Thm 1 no  & $9$ & $39$ \\\hline                              
    \end{tabular}
%    \caption{Squarefree isogeny classes of dimension $4$. The notation is the same as in Table \ref{tab:dim2}.}
    \label{tab:dim4}
\end{table}
\pause
{\small
	Thm 1 ($S^*_\R\subseteq T^*_\R$) doesn't handle $72/\F_2$ and $391/\F_3$.
\pause	
	Out of these, we can use other techniques for $20/\F_2$ and $214/\F_3$.
\pause
	For the remaining $52/\F_2$ and $171/\F_3$ we can only get information about certain endomorphism rings ($723$ out of $946$ and $3481$ out of $4636$, respectively).
% \pause	
% 	Also there are $9/\F_3$ for which the computations of the isomorphism classes of unpolarized abelian varieties is not over yet.
}
\end{frame}

\begin{frame}[noframenumbering]{ }
	\begin{center}
		\green{\huge Thank you!}
	\end{center}
\end{frame}

\begin{frame}[noframenumbering]{ Effective Results II }
	\begin{theorem}[2]
    Assume that there are $r$ isomorphism classes of abelian varieties in $\AV_h(p)$ with endomorphism ring $T$, represented under $\cG$ by the fractional ideals  $I_1,\ldots,I_r$.
    For any CM-type $\Phi'$, we put
		\[ \Pone{\Phi'}{I_i}=\{ i_0 \cdot u: u \in \mathcal{T} \text{ such that } i_0 u \text{ is totally imaginary and $\Phi'$-positive } \}. \]
    If there exists a non-negative integer $N$ such that for every CM-type $\Phi'$ we have
    \[
    \vert \Pone{\Phi'}{I_1} \vert + \ldots + \vert \Pone{\Phi'}{I_r} \vert = N
    \]
    then there are exactly $N$ isomorphism classes of principally polarized abelian varieties with endomorphism ring $T$. 
	\end{theorem}
\end{frame}

\begin{frame}[noframenumbering]{ }
    \begin{proof}
    \begin{itemize}
    \item Consider the association $\Phi'\mapsto b$ where $b\in L^*$ is tot.~imaginary and $\Phi'$-positive.
    \item We can go back: for every $b$ tot.~imaginary there exists a unique CM-type $\Phi_b$ s.t.~$b$ is $\Phi_b$-positive.
    \item Hence the totally real elements of $L^*$ acts on the set of CM-types.
    \item If $\Phi=\Phi_{b}$ is the CM-type for which we have a canonical lift (as before)
        then $\Palpha{\Phi_b}{I_i} \longleftrightarrow \Pone{\Phi_{\alpha b}}{I_i}$.
    \item If the we get the 'same sum' (over the $I_i$'s) for every CM-type we know that the result must be the correct one! 
    \end{itemize}
    \end{proof} 
    Note: even if the sum is not the same for all $\Phi'$'s then we know that one of the outputs is the correct one!
\end{frame}

\end{document}
