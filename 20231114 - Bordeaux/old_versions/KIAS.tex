%\documentclass[usenames,dvipsnames]{beamer}
\documentclass[usenames,dvipsnames,handout]{beamer}

\usetheme{AnnArbor}
% \usecolortheme{default}
% \usecolortheme{crane}
\usecolortheme{beaver}
\usecolortheme{dolphin}
% \usecolortheme{orchid}
% \usecolortheme{rose}


\usepackage{fourier}
\usepackage{faktor}
\usepackage{amssymb}
\usepackage{amsmath}
\usepackage{amsthm}
%\usepackage{stmaryrd}
\usepackage{hyperref}
\usepackage[all]{xy}
\usepackage{tikz}
%    \usetikzlibrary{mindmap,shadows,shapes.geometric,shapes.misc,positioning}
\usepackage{tikz-cd}
\tikzset{
    invisible/.style={opacity=0},
    visible on/.style={alt={#1{}{invisible}}},
    alt/.code args={<#1>#2#3}{%
      \alt<#1>{\pgfkeysalso{#2}}{\pgfkeysalso{#3}}%
  }
}
%\usetikzlibrary{matrix}
%\usetikzlibrary{calc,intersections}
%\newcommand{\downmapsto}{\rotatebox[origin=c]{-90}{$\large\mapsto$}\mkern2mu} %MnSymbol doesn't work well with beamer
\usepackage{multirow}

\def\Q{\mathbb{Q}}
\def\Z{\mathbb{Z}}
\def\C{\mathbb{C}}
\def\R{\mathbb{R}}
\def\F{\mathbb{F}}

\DeclareMathOperator{\AV}{AV}
\DeclareMathOperator{\Mat}{Mat}
\DeclareMathOperator{\Pol}{Pol}
\DeclareMathOperator{\Char}{char}
\DeclareMathOperator{\rk}{Rank}
\DeclareMathOperator{\Frob}{Frob}
\DeclareMathOperator{\ICM}{ICM}
\DeclareMathOperator{\Pic}{Pic}
\DeclareMathOperator{\Aut}{Aut}
\DeclareMathOperator{\Hom}{Hom}
\DeclareMathOperator{\End}{End}
\DeclareMathOperator{\Gal}{Gal}
\DeclareMathOperator{\mSpec}{mSpec}
\DeclareMathOperator{\GL}{GL}
\DeclareMathOperator{\Tr}{Tr}
\DeclareMathOperator{\Jac}{Jac}
%\renewcommand{\char}{char} %CRASHES WITH beamer


\newcommand{\cG}{\mathcal{G}}
%\newcommand{\cB}{{\mathcal B}}
%\newcommand{\cC}{{\mathcal C}}
\newcommand{\cO}{{\mathcal O}}
\newcommand{\cH}{{\mathcal H}}
%\newcommand{\cM}{{\mathcal M}}
\newcommand{\cT}{{\mathcal T}}
%\newcommand{\cW}{{\mathcal W}}


\newcommand{\vphi}{\varphi}

\newcommand{\p}{{\mathfrak p}}
\newcommand{\frf}{{\mathfrak f}}

\newcommand{\set}[1]{\left\lbrace#1\right\rbrace }
\newcommand{\Span}[1]{\left<#1\right>}

%\newcommand{\AVord}[1]{\AV^{\text{ord}}({#1})}
%\newcommand{\Modord}[1]{\cM^{\text{ord}}({#1})}

%\newcommand{\AVcs}[1]{\AV^{\text{cs}}({#1})}
%\newcommand{\Modcs}[1]{\cM^{\text{cs}}({#1})}

\newcommand{\Acan}{\mathcal{A}_{\mathrm{can}}}
\newcommand{\AcanC}{A_{\mathrm{can}}}
\newcommand{\Palpha}[2]{\mathcal{P}^{\alpha}_{{#1}}({#2})}
\newcommand{\Pone}[2]{\mathcal{P}^{1}_{{#1}}({#2})}

\newcommand{\red}[1]{\textcolor{red}{#1}}
\newcommand{\blue}[1]{\textcolor{blue}{#1}}
\newcommand{\green}[1]{\textcolor{ForestGreen}{#1}}

\newtheorem{df}{Definition}[section]
\newtheorem{remark}[df]{Remark}
\newtheorem{prop}[df]{Proposition}
\newtheorem{cor}[df]{Corollary}



%AUTHOR DETAILS
%%%%%%%%%%%%%%%%%%%%%%%%%%%%%%%%%%%%%%%%%%%%%%%%
\title[]{Polarizations of abelian varieties over finite fields\\ via canonical liftings}
\subtitle{}
\author[Stefano Marseglia]{Stefano Marseglia\\}
\institute[]{Utrecht University}
\date[12 May 2022]{KIAS Number Theory Seminar - 12 May 2022\\ \pause joint work with\\ {\bf Jonas Bergstr\"om} and {\bf Valentijn Karemaker}.}

\begin{document}

\begin{frame}
\titlepage
\end{frame}


\begin{frame}{ Abelian Varieties }
	\begin{itemize}
	 \item An {\bf abelian variety} $A$ over a field $k$ is a projective geometrically connected group variety over $k$.\\
	 \pause We have {\bf morphisms} $\oplus:A\times A \to A$, $\ominus:A\to A$ and a $k$-rational point $e\in A(k)$ such that $(A,\oplus,\ominus,e)$ is a group object in the category of projective geom.~connected varieties over $k$.
	 \pause \item In practice, we have {diagrams $\rightsquigarrow$} {\bf ``natural'' group structure} on $A(\overline k)$.
	 \pause \item eg. ($\ominus$ is the ``inverse'' morphism)
	 {\tiny
	 \[ 
	 	\xymatrix{
	 		A\times_k A \ar[rr]^{(\ominus,\mathrm{id})} 	& 						& A\times_k A \ar[d]^{\oplus}\\
	 		A \ar[u]^{\Delta} \ar[r] 	& \mathrm{Spec}(k) \ar[r]^{e}	& A 
	 	}
	 	\pause \qquad
	 	\xymatrix{
	 		A\times_k A \ar[rr]^{(\mathrm{id},\ominus)} 	& 						& A\times_k A \ar[d]^{\oplus}\\
	 		A \ar[u]^{\Delta} \ar[r] 	& \mathrm{Spec}(k) \ar[r]^{e}	& A 
	 	}
	  \]}
	\end{itemize}
\end{frame}

\begin{frame}{ Example : $\dim A=1$ elliptic curves  }
	\begin{itemize}
		\item AVs of dimension $1$ are called {\bf elliptic curves}.
		\pause \item They admit a {plane model}: if $\Char k \neq 2,3$
		\[  Y^2Z = X^3 +AXZ^2 + BZ^3\quad A,B \in k\text{ and }e=[0:1:0]  \]
		\pause \vspace{-0.7cm} \item The {groups law is explicit}:\\
		if $P=(x_P,y_P) $ then $ \ominus P=(x_P,-y_P) $ and\\
		\pause if $Q=(x_Q,y_Q)\neq \ominus P$ then $P\oplus Q=(x_R,y_R)$ where
		\[ x_R = \lambda^2 -x_P-x_Q, \quad y_R = y_P+\lambda (x_R-x_P), \]
		where
		\[ \lambda = 
		\begin{cases}
			\frac{3x_P^2 + B}{2A}& \text{ if } P=Q  \\
			\frac{y_P -y_Q}{x_P - x_Q} &  \text{ if } P\neq Q
		\end{cases}
		\]	
	\end{itemize}
\end{frame}

\begin{frame}{ Example : EC over $\R$ }
\vspace{-0.5cm}
\begin{center}
	\begin{columns}
		\begin{column}{0.4\textwidth}
			\onslide<2-> {Over $\R$:\\ 
			consider the abelian variety: 
			\[ \red{y^2 = x^3 -x +1} \]}
			\onslide<3->{Addition law: $P,Q$ {\Large \blue{$\leadsto$}} $P\oplus Q$}
		\end{column}
		\begin{column}{0.5\textwidth}
			\tikz{
			    \onslide<2->{
				\draw [help lines] (-2,-2.24) grid (1.8,2.24);
				\draw [->] (-2-0.2,0) -- (1.8+0.2,0) node[right] {$x$}; 
				\draw [->] (0,-2.24-0.2) -- (0,2.24+0.2) node[left] {$y$}; 
				\draw [red, thick, domain=-1.32471795724474602596090885448:1.8, samples=100]
				 plot (\x, {sqrt(\x*\x*\x -\x +1)});
				\draw [red, thick, domain=-1.32471795724474602596090885448:1.8, samples=100]
				 plot (\x, -{sqrt(\x*\x*\x -\x +1)});
				}
				\onslide<3->{
				\draw [blue, thick, domain=-2:1.8, samples=100] plot (\x, {0.7*\x +0.5 });
				\draw [blue, thick, dashed] (1.3407,2.24) -- (1.3407,-2.24);
				\draw (-1.2858-0.2,-0.7*1.2858+0.5+0.1) node {$P$};
				\draw (0.43506,0.7*0.43506+0.5 +0.3) node {$Q$};
				\draw (1.3407+0.5,-0.7*1.3407-0.5) node {$P\oplus Q$};
				}
			}
		\end{column}
	\end{columns}
\end{center}
\end{frame}

\begin{frame}{ Duals and Polarizations }
    \begin{itemize}
     \item A hom.~$\vphi:A\to B$ is an {\bf isogeny} if $\dim A = \dim B$ and $\vphi$ is surjective.
     \pause \item Isogenies have finite kernel:  $\deg\vphi = \mathrm{rank}(\ker(\vphi))$
	 \pause \item $\Pic^0_{A}$ is also an AV, called the {\bf dual} of $A$ and denoted $A^\vee$.	 
	 \pause \item An isogeny $\mu:A\to A^\vee$ (over $k$) is called a {\bf polarization} if 
     there are an $k\subseteq k'$ and an ample line bundle $\mathcal{L}$ such that (on points)
     \[ \vphi_{k'}: x\mapsto [t_x^*\mathcal{L} \otimes\mathcal{L}^{-1} ]. \]
     \vspace{-1.5em}
	 \pause \item A polarization $\mu$ is {\bf principal} if $\deg \mu = 1 \iff \mu$ is an isomorphism.
	 \pause \item \red{Why} do we care about polarizations?
        \begin{enumerate}
    	 \pause \item $\Aut(A,\mu)$ is finite $\leadsto$ moduli space $\mathcal{A}_{g,d}$
    	 \pause \item proper smooth curve $C/k \leadsto \Pic^0_{C}=:\Jac(C)$ a PPAV.
    	\end{enumerate}
	\end{itemize}
\end{frame}


\begin{frame}{ $\C$ vs $\F_q$ }    
    \begin{itemize}
     \item Pick $A/\C$ of dimension $g$. 
	 \pause \item $A(\C)\simeq V:=\C^g/\Lambda$, where $\Lambda\simeq_\Z\Z^{2g}$. It is a {torus}.
	 \pause \item $V$ admits a non-degenerate {\bf Riemann form} $\longleftrightarrow$ polarization.
	 \pause \item Actually,
	  \[
      \set{ \text{abelian varieties $/\C$} } \longleftrightarrow 
      \set{\parbox[c]{12.5em}{\center $\C^g/\Lambda$ with $\Lambda\simeq \Z^{2g}$ admitting\\ a Riemann form}}
     \]
	  induced by $ A \mapsto A(\C)$ is an \red{equivalence} of categories.
	 \pause \item In {char.~$p>0$} such an equivalence {cannot exist} : there are (supersingular) elliptic curves with quaternionic endomorphism algebras.
	\end{itemize}
\end{frame}

\begin{frame}{ Canonical Liftings } 
    \begin{itemize}
    \item Let $A_0$ be an abelian variety over $\F_q$ of dim $g$.
\pause
    \begin{definition}
	    A {\bf canonical lifting} of $A_0$ is an abelian scheme over a normal local domain $\mathcal{R}$ of characteristic zero with residue field $\F_q$ with:
	    \begin{enumerate}
	    \item special fiber $A_0$, and
	    \item general fiber $\Acan$ satisfying $\End(\Acan)= \End(A_0)$.
	    \end{enumerate}
    \end{definition}
\pause
    \item $A_0$ comes with a Frobenius endomorphism induced by $x\mapsto x^q$ on coordinates rings (we are in $\mathrm{char}(\F_q) =p>0$!)
\pause    
    \item Example: ordinary abelian variety; almost-ordinary abelian variety (with commutative $\F_q$-endomorphism algebra). 
\pause
    \item Non-example: supersingular EC with quaternionic end.~algebra.
	\end{itemize}
\end{frame}

\begin{frame}{ Complex Uniformization }
    \begin{itemize}
    \item Assume that $A_0$ admits a canonical lifting $\Acan$.
\pause   
    \item Fix $\mathcal{R} \hookrightarrow \C$ and put $\AcanC:=\Acan \otimes \C$.
\pause
    \item $\AcanC$ has morphisms $F$ (and $V=\frac{q}{F}$) reducing to Frobenius (and Verschiebung).
\pause
    \item By {\bf complex uniformization}:
    \[ \AcanC(\C) \simeq \faktor{\C^g}{\Phi(I)} \quad 
    \parbox{18em}{ - $I$ : a fractional $\Z[F,V]$-ideal in $L:=\Q[F]$,  \\
                  - $\Phi$ : a {\bf CM-type} of $L$ ($g$ maps $L\to \C$, one per conjugate pair). } \]
\pause
    \vspace{-1.3em}
    \item Define $\cH(\AcanC):=I$.
\pause
    \item By the same construction:
    \[
    \begin{tikzcd}[ampersand replacement=\&] %needed for Beamer
    \text{char.}0 : \& \Acan^\vee \arrow[r, "\otimes \C"] \& \AcanC^\vee \arrow[r, "\mathcal{H}"] \& \overline{I}^t =\set{ \overline{x} : \Tr_{L/\Q}(xI)\subseteq \Z }\\
    \F_q : A_0^\vee \arrow[ur] \& \& \&
    \end{tikzcd}    
    \]
\pause
    \vspace{-1.5em}
    \item In particular: $\cH(\Hom(\AcanC,\AcanC^\vee)) = (\overline{I}^t:I) = \set{ x \in L : xI \subseteq \overline{I}^t }$.
	\end{itemize}
\end{frame}

\begin{frame}{ Complex Uniformization : Polarizations }
    \begin{itemize}
    \item We have: 
    \[ \AcanC(\C) \simeq \faktor{\C^g}{\Phi(I)}, \quad \AcanC^\vee(\C) \simeq \faktor{\C^g}{\Phi(\overline{I}^t)},\]
    \[ \cH(\Hom(\AcanC,\AcanC^\vee)) = (\overline{I}^t:I). \]
\pause         
    \item What about {\bf polarizations}? We understand them over $\C$!
\pause    
    \item Let $\mu:\AcanC\to \AcanC^\vee$ an isogeny. Then $\mu$ is a polarization if and only if
	       $\lambda := \cH(\mu) \in (\overline{I}^t:I)$ satisfies
	       \begin{enumerate}
\pause 
	       \item $\lambda = - \overline{\lambda}$ (\blue{totally imaginary}), and
\pause 
	       \item for every $\vphi\in \Phi$ we have $Im(\vphi(\lambda))>0$  (\blue{$\Phi$-positive}).
           \end{enumerate}
\pause            	  
	\end{itemize}
\end{frame}


\begin{frame}{ Isogeny classification over $\F_q$}
	\begin{itemize}
    \item The {Frobenius endomorphism} $A/\F_{q}$ comes 
    \pause induces an action
		\[ \Frob_A : T_\ell A \rightarrow T_\ell A \text{ for any }\ell\neq p, \]
		where $T_\ell(A) = \varprojlim A[\ell^n] \simeq \Z_\ell^{2g}$.
	\pause \item ${h_A(x)}:=\Char(\Frob_A)$ is a {\bf $q$-Weil} polynomial and isogeny {invariant}.
	\pause \item By \red{Honda-Tate} theory, the association
		\[ A \longmapsto h_A(x) \]
		is injective up-to-isogeny and allows us to \red{list} all isogeny classes.
	\pause \item One can prove that $h_A(x)$ is squarefree $\iff$ $\End(A)$ is commutative.
	\end{itemize}
\end{frame}

\begin{frame}{ Isomorphism classification over $\F_p$ }
\pause 
    \begin{theorem}[Centeleghe-Stix]
    Let $\AV_h(p)$ be the isogeny class over the {\bf prime field} $\F_p$ determined by a {\bf squarefree} characteristic polynomial of Frobenius $h$.\\
    Let $L=\Q[x]/h=\Q[F]$ be the endomorphism algebra, and put $V=p/F$.\\
\pause 
    There is an \red{equivalence} of categories:
    \[ \AV_h(p) \overset{\cG}{\longrightarrow} \set{\text{fractional $\Z[F,V]$-ideals in $L$}}.  \]
    \end{theorem}
	\begin{itemize}
\pause 
    \item Let $A_h$ be an AV in $\AV_h(p)$ with $\End(A_h)=\Z[F,V]$.
\pause 
    \item The functor $\cG(-):=\Hom(-,A_h)$ induces the equivalence.
\pause 
    \item We can {\bf choose} $A_h$ so that for every $B_0\in \AV_h(p)$:
    \[ \blue{\cG(B_0^\vee) = \overline{\cG(B_0)}^t}
\pause    
     \text{ and } \blue{\cG(f^\vee) = \overline{\cG(f)}}, \text{ for any }f: B_0\to B_0' \text{ in }\AV_h(p). \]
\pause
    \vspace{-0.8cm}     
    \item In particular:
    \vspace{-0.5cm}    
    \[ \cG(\Hom(B_0,B_0^\vee)) = (\cG(B_0):\overline{\cG(B_0)}^t). \]
	\end{itemize}
\end{frame}

\begin{frame}{ Comparison }
    \begin{itemize}
    \item Assume that $A_0$ admits a canonical lifting $\AcanC$.
    \item We have two description using fractional ideals. Let's compare them.
\onslide<5->{\item Let $f:A_0\to B_0$ be an isogeny.}
    \end{itemize} 
    \[
%    \small
    \begin{tikzcd}[ampersand replacement=\&] %needed for Beamer
    \& |[visible on=<2->]| \Hom(\AcanC,\AcanC^\vee) \arrow[visible on=<2->,d, "\mathrm{red}"]    \arrow[visible on=<2->,dr, "\text{complex unif.}"] \& \\
    |[visible on=<5->]|\Hom(B_0,B_0^\vee) \arrow[visible on=<5->,r,"f^*:=f^\vee\circ - \circ f"] \arrow[visible on=<6->,d, "\cG"] \& 
    |[visible on=<2->]| \Hom(A_0,A_0^\vee)\arrow[visible on=<3->,d,"\cG"] \&
    |[visible on=<2->]| (\overline I^t : I) 
    \arrow[visible on=<3->]{d}[visible on=<4->,swap]{\red{\cdot\alpha}}[visible on=<4->,right,near start]{\red{\text{tot.~real }(\alpha=\overline{\alpha})}}  [visible on=<4->,right,near end]{\red{\text{unit in }\End(A_0)}}\\
	|[visible on=<6->]| (\cG(B_0) : \overline{\cG(B_0)}^t)\arrow[visible on=<6->,r,"\cG(f^*)"] \&
	|[visible on=<3->]| (\cG(A_0) : \overline{\cG(A_0)}^t) \& |[visible on=<3->]| (\overline I^t : I)
	\arrow[visible on=<3->,l,equal]
	\end{tikzcd}
	\]
\onslide<6->
    \begin{itemize}
    \item $f^*$ sends polarizations to polarizations.
\onslide<7->    
    \item $\cG(f^*) = \overline{\cG(f)}\cG(f)$ is a totally positive element:\\
\onslide<8->
          it sends totally imaginary elements to totally imaginary elements and $\Phi$-positive elements to $\Phi$-positive elements.
    \end{itemize}     
\end{frame}

\begin{frame}{ Comparison : Polarizations }
\onslide<1->
    \[
    \begin{tikzcd}[ampersand replacement=\&] %needed for Beamer
    \&[-35pt] |[visible on=<2->]| \parbox{3cm}{\small\centering\red{we WANT \\to understand\\pols.~here}} \arrow[visible on=<2->,d, red]  \& \Hom(\AcanC,\AcanC^\vee) \arrow[d, "\mathrm{red}"] \arrow[dr] \&[-15pt] 
    |[visible on=<2->]| \parbox{1.5cm}{\small\centering\green{we DO\\ understand\\pols.~here}} \arrow[visible on=<2->,d,ForestGreen]\&[-43pt]\\[-15pt]
    |[visible on=<3->]| \red{\mu \in} \&[-35pt] 
    \Hom(B_0,B_0^\vee) \arrow[r,"f^*:=f^\vee\circ - \circ f"] \arrow[d, "\cG"] \& 
    \Hom(A_0,A_0^\vee)\arrow[d,"\cG"] \&[-35pt]
    (\overline I^t : I)
    \arrow{d}[swap]{\cdot\alpha} \&[-25pt] 
    |[visible on=<3->]| \green{\ni \alpha^{-1}\cG(f^*)\cG(\mu)} \\
	\&[-35pt] (\cG(B_0) : \overline{\cG(B_0)}^t)\arrow[r,"\cG(f^*)"] \&
	(\cG(A_0) : \overline{\cG(A_0)}^t) \&[-35pt] (\overline I^t : I)
	\arrow[l,equal]  \&[-35pt] 
	\end{tikzcd}
	\]
\onslide<3->
    By chasing the diagram, we get:
\onslide<4->
	\begin{theorem}["lift and spread"]
	Let $\mu:B_0 \to B_0^\vee$ be an isogeny. Then
    \[ \mu \text{ is a \red{polarization}} \iff \alpha^{-1} \cG(\mu) \text{ \green{is totally imaginary} and \green{$\Phi$-positive}} \]
	\end{theorem}
\end{frame}

\begin{frame}{ Principal Polarizations up to isomorphism }
    \begin{itemize}
	 \item Let $B_0 \in \AV_h(p)$. Put $T=\End(B_0)$ and $\cG(B_0)=J$.
	 \pause \item Assume that $B_0 \simeq B_0^\vee$, i.e.~$J=i_0\overline{J}^t$ for some $i_0\in L^*$.
	 \pause \item If $\mu$ and $\mu'$ are principal polarizations of $B_0$ then $(B_0,\mu) \simeq (B_0,\mu')$ (as PPAVs) if and only if there is $v\in T^*$ such that $\cG(\mu)=v\overline{v}\cG(\mu')$.
	 \pause \item Let $\cT$ be a transversal of $T^*/<v\overline{v} : v \in T^*>$.
	 \pause \item Then 
	 \[ \Palpha{\Phi}{J}:=\{ i_0 \cdot u  : u \in \mathcal{T} \text{ s.t.~} \alpha^{-1} i_0 u \text{ is tot.~imaginary and } \Phi\text{-positive} \} \]
	 is a set or representatives of the PPs of $B_0$ up to isomorphism.
	 \pause \item It depends on $\alpha$!
	\end{itemize}
\end{frame}

\begin{frame}{ Effective Results : when can we ignore $\alpha$? }
    Assume $A_0$ admits a canonical lifting. Put $S:=\End(A_0)$\\
    Let $B_0$ be isogenous to $A_0$. Put $T=\End(B_0)$.
\pause      
	\begin{theorem}[ 1 ]
		Denote by $S^*_\R$ (resp.~$T^*_\R$) the group of totally real units of $S$ (resp.~$T$).\\
\pause 
		If $S^*_\R\subseteq T^*_\R$, then the set
		\[ \Palpha{\Phi}{J}:=\{ i_0 \cdot u  : u \in \mathcal{T} \text{ s.t.~} \alpha^{-1} i_0 u \text{ is tot.~imaginary and } \Phi\text{-positive} \} \]
\pause 
	    is in bijection with the set (which does not depend on $\alpha$!)
        \[ \Pone{\Phi}{J}=\{ i_0 \cdot u: u \in \mathcal{T} \text{ such that } i_0 u \text{ is totally imaginary and $\Phi$-positive } \}. \]
	\end{theorem}
\pause 
	\begin{cor}
    If $S=\Z[F,V]$ (eg. $\AV_h(p)$ is ordinary or almost-ordinary) then we can ignore $\alpha$.
\pause 
    \green{We recover Deligne+Howe and Oswal-Shankar}
	\end{cor}
\end{frame}


\begin{frame}
	We run computations over all squarefree isogeny classes over small prime fields of dim $2,3$ and $4$. 
	\pause For example:
	\begin{table}[ht]
	    \centering
	    \small
	    \begin{tabular}{|c|c|c|c|c|c|c|}\hline
	\multicolumn{3}{|c|}{squarefree dimension $3$}                  & $p=2$ & $p=3$ & $p=5$ & $p=7$ \\\hline
	\multicolumn{3}{|c|}{total}                                     & $185$ & $621$  & $2863$ & $7847$ \\\hline                    
	\multicolumn{3}{|c|}{ordinary}                                  & $82$ & $390$  & $2280$  & $6700$ \\\hline
	\multicolumn{3}{|c|}{almost ordinary}                           & $58$ & $170$  & $474$  & $996$  \\\hline
	\multirow{3}{*}{$p$-rank $1$} & \multicolumn{2}{|c|}{cannot lift}    & $0$ & $0$   & $0$   & $0$   \\\cline{2-7}
	                              & \multirow{2}{*}{can lift} & Thm 1 yes & $20$ & $26$   & $76$   & $118$   \\\cline{3-7}
	                              &                          & Thm 1 no  & $4$ & $16$   & $12$   & $8$   \\\hline
	\multirow{3}{*}{$p$-rank $0$}   & \multicolumn{2}{|c|}{cannot lift}    & $0$ & $3$   & $2$   & $1$   \\\cline{2-7}
	                              & \multirow{2}{*}{can lift} & Thm 1 yes & $20$ & $15$   & $17$   & $23$   \\\cline{3-7}
	                              &                          & Thm 1 no  & $1$ & $1$   & $2$   & $1$   \\\hline                              
	    \end{tabular}
	%    \caption{Squarefree isogeny classes of dimension $3$. The notation is the same as in Table \ref{tab:dim2}.}
	%    \label{tab:dim3}
	\end{table}
    \pause Among the $45$ isogeny classes which we cannot 'handle' with Thm 1, we can compute the number of PPAV for $32$ of them using Thm 2. For the remaining $13$ (all over $\F_2$ and $\F_3$) we only get partial info.
\end{frame}

\begin{frame}{  }
    \begin{center}
    \green{\huge Thank you!}
    \end{center}  
\end{frame}

% SECRET SLIDES

\begin{frame}[noframenumbering]{ Effective Results II }
	\begin{theorem}[2]
    Assume that there are $r$ isomorphism classes of abelian varieties in $\AV_h(p)$ with endomorphism ring $T$, represented under $\cG$ by the fractional ideals  $I_1,\ldots,I_r$.
    For any CM-type $\Phi'$, we put
		\[ \Pone{\Phi'}{I_i}=\{ i_0 \cdot u: u \in \mathcal{T} \text{ such that } i_0 u \text{ is totally imaginary and $\Phi'$-positive } \}. \]
    If there exists a non-negative integer $N$ such that for every CM-type $\Phi'$ we have
    \[
    \vert \Pone{\Phi'}{I_1} \vert + \ldots + \vert \Pone{\Phi'}{I_r} \vert = N
    \]
    then there are exactly $N$ isomorphism classes of principally polarized abelian varieties with endomorphism ring $T$. 
	\end{theorem}
\end{frame}

\begin{frame}[noframenumbering]{ Effective Results II }
    \begin{proof}
    \begin{itemize}
    \item Consider the association $\Phi'\mapsto b$ where $b\in L^*$ is tot.~imaginary and $\Phi'$-positive.
    \item We can go back: for every $b$ tot.~imaginary there exists a unique CM-type $\Phi_b$ s.t.~$b$ is $\Phi_b$-positive.
    \item Hence the totally real elements of $L^*$ acts on the set of CM-types.
    \item If $\Phi=\Phi_{b}$ is the CM-type for which we have a canonical lift (as before)
        then $\Palpha{\Phi_b}{I_i} \longleftrightarrow \Pone{\Phi_{\alpha b}}{I_i}$.
    \item If the we get the 'same sum' (over the $I_i$'s) for every CM-type we know that the result must be the correct one! 
    \end{itemize}
    \end{proof} 
    Note: even if the sum is not the same for all $\Phi'$'s then we know that one of the outputs is the correct one!
\end{frame}


\begin{frame}[noframenumbering]{ When can we lift up to isogeny? }
	\begin{df}[Chai-Conrad-Oort]
		Let $\Phi$ be a $p$-adic CM-type for a CM-field $L=\Q(F)$.
		\pause The pair $(L,\Phi)$ satisfies the \red{Residual Reflex Condition} w.r.t.~$F$ 
		if the following conditions are met:
		\begin{enumerate}[1.]
			\pause \item \label{def:RRC_item_st} The {\bf Shimura-Taniyama formula} holds for $F$: for every  place $\nu$ of $L$ above~$p$, we have
			\begin{equation*}
			\dfrac{ \mathrm{ord}_\nu(F)}{ \mathrm{ord}_\nu(q)}=\dfrac{\#\set{ \vphi \in \Phi \text{ s.t.~} \vphi \text{ induces } \nu }}{[L_\nu:\Q_p]}.
			\end{equation*}
			\pause \item \label{def:RRC_item_refl} Let $E$ be the reflex field attached to $(L,\Phi)$, and let $\nu$ be the induced $p$-adic place of $E$. Then the {\bf residue field} $k_\nu$ of $\cO_{E,\nu}$
			can be realized as a {\bf subfield} of $\F_q$.
		\end{enumerate}
	\end{df}
\end{frame}

\begin{frame}[noframenumbering]{ When can we lift up to isogeny?  }
	\begin{theorem}[Chai-Conrad-Oort]
    Assume that $(L,\Phi)$ satisfies the {\bf Residual Reflex Condition} w.r.t.~$F$, that is,\\
    \begin{enumerate}
    \item $\Phi$ satisfies the {Shimura-Taniyama} formula for $F$, and
    \item the reflex field $E$ has residue field {$k_E\subseteq \F_q$}.
    \end{enumerate}  
\pause 
    Then we can \red{canonically lift} an abelian variety $A_0$ with $\mathcal{O}_L = \mathrm{End}(A_0)$.
	\end{theorem}
	\begin{itemize}
%\pause 
%    \item Can generalize: $h$ irreducible $\leadsto$ $h$ squarefree.
\pause
    \item If there is a separable isogeny $A_0\to A'_0$ then $A'_0$ admits a canonical lifting (useful in combination with Thm 1).
	\end{itemize}
\end{frame}


\begin{frame}[noframenumbering]{}{}
\begin{table}[ht]
    \centering
\footnotesize	
    \begin{tabular}{|c|c|c|c|c|}\hline
\multicolumn{3}{|c|}{squarefree dimension $4$}                  & $p=2$ & $p=3$ \\\hline
\multicolumn{3}{|c|}{total}                                     & $1431$ & $10453$  \\\hline                    
\multicolumn{3}{|c|}{ordinary}                                  & $656$ & $6742$  \\\hline
\multicolumn{3}{|c|}{almost ordinary}                           & $392$ & $2506$  \\\hline
\multirow{3}{*}{$p$-rank $2$} & \multicolumn{2}{|c|}{cannot lift}    & $0$ & $0$ \\\cline{2-5}
                              & \multirow{2}{*}{can lift} & Thm 1 yes & $149$ & $500$   \\\cline{3-5}
                              &                          & Thm 1 no  & $49$ & $312$   \\\hline
\multirow{3}{*}{$p$-rank $1$} & \multicolumn{2}{|c|}{cannot lift}    & $6$ & $36$ \\\cline{2-5}
                              & \multirow{2}{*}{can lift} & Thm 1 yes & $80$ & $184$   \\\cline{3-5}
                              &                          & Thm 1 no  & $14$ & $40$   \\\hline
\multirow{3}{*}{$p$-rank $0$} & \multicolumn{2}{|c|}{cannot lift}    & $3$ & $6$   \\\cline{2-5}
                              & \multirow{2}{*}{can lift} & Thm 1 yes & $73$ & $88$ \\\cline{3-5}
                              &                          & Thm 1 no  & $9$ & $39$ \\\hline                              
    \end{tabular}
%    \caption{Squarefree isogeny classes of dimension $4$. The notation is the same as in Table \ref{tab:dim2}.}
    \label{tab:dim4}
\end{table}
{\small
	Thm 1 ($S^*_\R\subseteq T^*_\R$) doesn't handle $72/\F_2$ and $391/\F_3$.
	Out of these, we can use Thm 2 for $20/\F_2$ and $214/\F_3$.
	For the remaining $52/\F_2$ and $171/\F_3$ we can only get information about certain endomorphism rings ($723$ out of $946$ and $3481$ out of $4636$, respectively).
	Also there are $9/\F_3$ for which the computations of the isomorphism classes of unpolarized abelian varieties is not over yet.
}
\end{frame}

\end{document}
